\section{Suicidio}

Dal punto di vista psichiatrico, il \textbf{\emph{suicidio}} può essere
considerato come la \emph{risposta dell'uomo alla mancanza di
significato del suo vivere}, essendo dettato da diversi moti emotivi
coscienti ed incoscienti, ed esprime una fuga dall'angoscia, ritenuta
intollerabile.
\\\\
Il suicidio è un'\emph{urgenza psichiatrica}, per cui da un punto di
vista medico l'obiettivo consiste nella riduzione o nella rimozione
delle cause e nel creare delle prospettive alternative di fuga, sia
questa reale o psicologica.
\\\\
Il suicidio costituisce quindi la \emph{conclusione di numerosi disturbi
psichiatrici}, ma quello in cui si osserva più comunemente è la
\textbf{depressione}, infatti il paziente depresso si suicida nel
tentativo di liberarsi da una condizione che sente altrimenti di non
poter modificare, cioè è convinto di non poter guarire e nella morte
pensa di trovare l'unica cosa che possa porre fine alla propria
sofferenza. Inoltre, in alcuni casi (soprattutto quando nella
depressione predominano dei forti sensi di colpa), il suicidio può
assumere una connotato di espiazione, come una sorta di
\emph{purificazione e riparazione per le proprie colpe}. Il
\textbf{\emph{tentativo di suicidio}} compare in genere nella \emph{fase
di remissione iniziale}, quando permane ancora una forte angoscia, ma il
paziente recupera un discreto grado di pianificazione ed ideazione, per
cui soprattutto in questa fase non si deve tralasciare il rischio di
suicidio, soprattutto se sono presenti degli elementi predisponenti come
i \emph{sentimenti di colpa}, l'\emph{ipocondria grave}, la
\emph{perdita di controllo} o i \emph{precedenti tentativi di suicidio}.
Altri elementi che possono giocare un ruolo nella predisposizione al
suicidio sono poi le \emph{incomprensioni da parte della famiglia}, la
\emph{presenza di una grave patologia organica} e le \emph{difficoltà
finanziarie e sociali rilevanti}.
\\\\
Negli \textbf{schizofrenici}, invece, il tentativo di suicidio è un
evento di difficile previsione, ed è del tutto \emph{incomprensibile},
attuandosi peraltro con \emph{modalità inconsuete}, manifestandosi
principalmente nella fase iniziale dell'acme dell'esperienza
catastrofica o nella fase di remissione, cioè quando il paziente diventa
in parte consapevole della propria condizione.
\\\\
In caso di tentativo di suicidio, inoltre, si devono stabilire le
\textbf{modalità} \textbf{adottate dal soggetto}, il \textbf{tipo di
eventuali strumenti lesivi usati}, la \textbf{gravità delle lesioni}, la
\textbf{minaccia di un'azione lesiva ritardata}, il \textbf{tipo di
paziente} ed i \textbf{precedenti psicopatologici}.
\\\\
Nella maggior parte dei casi si opta in genere per il \emph{ricovero
tempestivo}, anche se in altri casi si può optare per una \emph{terapia
a domicilio}, comunque con visite frequenti se il paziente è cooperante.
Pazienti molto a rischio sono poi gli \textbf{uomini} \textbf{anziani}
al di sopra dei 75 anni, in cui il suicidio fa spesso seguito ad un
evento di vita traumatico, che il paziente non riesce ad accettare o a
metabolizzare, anche se negli ultimi anni si è assistito ad un aumento
dell'incidenza del suicidio negli adolescenti.
\\\\
Esiste poi una classificazione del suicidio, proposta alla fine dell'800
da Emile Durkheim, in base alla quale si possono distinguere:

\begin{itemize}
\item
  \textbf{\emph{Suicidio Egoistico}}, cioè una condizione in cui
  l'individuo è scarsamente integrato nel contesto sociale, tende
  all'isolamento e quindi all'individuazione, e questo porta ad una
  condizione di egoismo affettivo.
\item
  \textbf{\emph{Suicidio Altruistico}}, che è l'opposto del suicidio
  egoistico, cioè in cui vi è una scarsa individualizzazione che porta
  poi all'identificazione nel gruppo.
\item
  \textbf{\emph{Suicidio Anomico}}, che si caratterizza per un
  cambiamento drastico, drammatico delle condizioni sociali, che può
  portare l'individuo a commettere suicidio, ad esempio a seguito di
  crolli economici e o dissesti finanziari importanti nell'ambito della
  società, e l'individuo si sente praticamente abbandonato dalla
  società.
\end{itemize}

È molto importante, inoltre, distinguere tra \textbf{tentativi di
suicidio completi}, in cui c'è l'intenzionalità a morire, e
\textbf{tentativi incompleti}, in cui si ricerca invece aiuto o
attenzione dagli altri; quest'ultimo tipo è caratteristico dei
\emph{disturbi di personalità di cluster B}, soprattutto il DP
borderline, mentre negli anziani sono più comuni i tentativi indiretti,
come il non assumere più i farmaci, o non seguire in maniera adeguata il
trattamento.

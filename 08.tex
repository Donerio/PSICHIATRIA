\section{Delirium}

\subsection{Definizione}

Il \textbf{\emph{delirium}} è una sindrome caratterizzata da una
\emph{compromissione dello stato di coscienza e delle funzioni
cognitive}, che da un punto di vista psichiatrico si associa anche a
vari \emph{disturbi del comportamento, dell'umore e della percezione},
per cui si tratta in realtà di un quadro composito che è la via finale
comune di tante noxe patogene, metaboliche o di tipo strutturale.

Tra i sintomi positivi, come le allucinazioni, si trova anche il
delirio: è un concetto molto complesso e specifico ed è un termine che
spesso viene utilizzato in maniera impropria. È classico, ad esempio,
che un paziente anziano con demenza e fenomeni di confabulazione venga
etichettato come ``delirante''.

Considerando l'etimologia della parola, deriva dal latino \emph{`de
lira'}: lira era propriamente il solco dell'aratro. Delirio perciò
indica l'uscita dal solco, inteso, fondamentalmente, come il solco in
cui sta il senso comune. Etimologicamente significherebbe uscire di
senno e già questo è piuttosto indicativo.

Infatti, paragonandolo al significato etimologico di ossessione (da
\emph{`obsidere'}=assediare) ci si accorge che sia, per certi aspetti,
l'esatto opposto del delirio: chi ha un disturbo ossessivo-compulsivo è
assediato da pensieri prodotti dalla sua mente, che riconosce come
morbosi. Il concetto di ossessione implica che il paziente sia normale,
ma si renda conto della presenza di qualcosa che lo assedi. Delirio,
invece, già nell'etimo indica qualcosa di più grave, perché non si è
assediati da pensieri, ma si esce dalla logica normale di pensare: tutti
ragionano e pensano in un determinato `solco', mentre il paziente
delirante sta in un solco parallelo.

Esistono due definizioni classiche di delirio:

\begin{itemize}
\item[1.]
  Definizione storica, che venne impiegata fino agli inizi del
  Novecento: \textbf{errore morboso di giudizio}. (\emph{Kraepelin})

Significa che vi è un errore nel giudizio enunciato dal paziente. Questa
definizione può essere applicata per alcune forme di delirio: ad
esempio, \emph{un paziente che riferisce di essere stato rapito dagli
alieni, oppure un paziente che dice di essere Gesù, oppure un paziente
che riferisce di essere controllato dalla CIA o dai servizi segreti
(delirio persecutorio)\ldots{}}è evidente che il giudizio sia errato,
non verosimile, assurdo. Chiunque, anche non uno psichiatra, sa che la
persona sta delirando.

In realtà questa è una definizione incompleta o addirittura non
corretta, che ha portato a molti equivoci: per anni il soggetto che
delira è stato considerato colui che per qualche motivo giudica male la
realtà, non riconoscendosi più nelle comuni connessioni logiche, ma tale
definizione implica in realtà la capacità di giudicare, l'intelligenza
che però nel delirante non è compromessa: il paziente vive qualcosa di
molto più profondo. Un soggetto può essere delirante perché
schizofrenico, o paranoide ad esempio, ma mantenere capacità
intellettive; intelligenza, memoria integre!

\item[2.]
  Definizione di Jaspers, formulata nel 1913: \textbf{giudizio
  patologicamente falsato}.

Apparentemente assomiglia alla precedente definizione, ma esiste una
differenza sostanziale. Qui il delirio è definito in base al MODO in cui
si giunge a quel giudizio. La definizione non poggia più sul contenuto
del delirio, ma sulla FORMA, sulla modalità formale che porta allo
strutturarsi di quel giudizio.

L'errore che spesso si fa tra i non specialisti è di concentrarsi sul
contenuto e non sulla forma. È ovvio che certi contenuti siano talmente
assurdi da indicare già di per sé un delirio, ma non è sempre così,
anche perché esistono deliri che esprimono contenuti verosimili.
Addirittura può capitare anche che alcuni contenuti di deliri siano
veri.

Ad esempio, un contenuto di delirio molto frequente e verosimile è il
\emph{delirio di gelosia: il paziente può sostenere di essere tradito
dalla moglie}. Anzi, può anche capitare di scoprire che questo accada
veramente. Che cosa, allora, rende questa persona delirante? La modalità
formale con cui si giunge a questo contenuto:

\begin{itemize}
\item
  \emph{paziente sostiene che la moglie lo tradisca e dice di saperlo
  con assoluta certezza perché ieri, uscendo di casa, ha visto un gatto
  nero passare e da lì l'ha capito} delirio di gelosia sorretto, sul
  piano formale, da una percezione delirante. Verosimilmente questo
  paziente apparterrà alla sfera schizofrenica, perché la percezione
  delirante è un sintomo di primo rango di Schneider.
\item
  \emph{paziente sostiene che la moglie lo tradisca e dice di saperlo
  con assoluta certezza perché afferma di non sentirsi più un uomo, di
  essere finito, di non portare a casa i soldi e quindi di meritare di
  essere tradito, mentre la moglie merita la felicità con un altro uomo}
  delirio affettivo, legato ad un disturbo depressivo, che viene detto
  olotimico, cioè legato all'affettività.
\item
  \emph{paziente sostiene che la moglie lo tradisca e giustifica questa
  convinzione dicendo che la moglie solitamente torni a casa alle 18.00,
  mentre ieri è tornata alle 18.02; ha guardato e calcolato la benzina
  consumata dall'automobile della moglie rapportandola con il tragitto
  per il lavoro}\ldots{} delirio di gelosia, con modalità formale su
  base interpretativa: il paziente interpreta tutti i dati di realtà. È
  la forma classica della paranoia.
\end{itemize}

Tre modalità formali differenti sostengono lo stesso contenuto, ma è in
base alla forma che facciamo sempre una diagnosi diversa.

\emph{Caso clinico:Si riporta l'esempio di una} \emph{paziente puerpera,
con deliri di colpa post-partum, per la quale il consulente si concentrò
sul contenuto di colpa e di rovina, dimettendola con affidamento ai
servizi sociali per risolvere la questione socio-economica di cui
parlava. Tuttavia, la paziente non aveva nessun problema di questo tipo,
ma si trattava di un delirio depressivo (con i classici contenuti di
colpa e rovina) e tentò il suicidio dopo la dimissione. }

Jaspers parlò di 3 caratteri fondamentali del delirio, che in ogni caso
devono essere considerati con le dovute cautele:

\begin{itemize}
\item
  \textbf{Assoluta certezza soggettiva}: il paziente è assolutamente
  convinto di ciò che dice;
\item
  \textbf{Non influenzabile e incorreggibile di fronte ad ogni
  confutazione logica}: con le parole il delirio non può recedere in
  alcun modo;
\item
  \textbf{Assurdità di contenuto}: questo criterio ovviamente è
  opinabile, perché è vero che talvolta i contenuti siano assurdi, ma
  esistono anche i casi di contenuti veri.
\end{itemize}

In realtà questi tre criteri, come tutte le regole, sono opinabili:
l'assurdità di contenuto ad esempio non caratterizza necessariamente un
delirio (caso classico è il delirio di gelosia: non è assurdo credere al
tradimento).

\end{itemize}

\subsection{Epidemiologia}

L'esordio è tipicamente molto \emph{rapido}, ed il decorso è
\emph{fluttuante}, ma breve. Fondamentale è l'identificazione del
fattore eziologico, perché trattando questo si elimina la
sintomatologia. Dal punto di vista strettamente epidemiologico, il
delirium è tuttavia spesso misdiagnosticato, in quanto confuso con
manifestazioni di alterato stato di coscienza meno gravi, ma si stima
che in ambito ortopedico fino al 30\% dei pazienti sottoposti ad
intervento chirurgico per la rottura della testa del femore sviluppa una
condizione di delirium, ed infatti quelle traumatiche sono tra le cause
più comuni di delirium, e questo è importante da tenere a mente perché
\emph{la contenizione non dovrebbe essere fatto in soggetto affetto da
delirium da cause organiche}, poiché in questo caso i sintomi tendono a
peggiorare anche considerevolmente, ed altre condizioni che lo possono
scatenare o aggravare sono lo \emph{stress} sia fisico che mentale, le
\emph{alterazioni del ritmo sonno-veglia}, il \emph{dolore intenso}, gli
\emph{stati tossinfettivi} e \emph{febbrili}, nonché le \emph{terapie}
stesse.

Il delirium, ovviamente, risulta più comune nelle fasce estreme della
vita, ovvero nei \textbf{bambini}, che risultano particolarmente
suscettibili alle patologie tossinfettive e febbrili, e gli
\textbf{anziani}, in cui la compromissione delle condizioni fisiche
generali si sommano ad un deterioramento delle funzioni cognitive,
predisponendo alle alterazioni dello stato di coscienza.

\subsection{Eziopatogenesi}

Le cause del delirium sono quindi molto numerose e variabili, e possono
essere generalmente suddivise in:

\begin{itemize}
\item
  \textbf{Cause Intracraniche}, che vanno dall'epilessia con stato di
  male epilettico sino a processi infettivi o a malattie vascolari;
\item
  \textbf{Cause Extracraniche}, tra cui si devono ricordare i farmaci ad
  azione anticolinergica, ma anche la cimetidina, alcuni antipertensivi
  ed antiepilettici, i salicilati, gli oppiacei ed i corticosteroidi, ma
  non si devono dimenticare anche condizioni come l'intossicazione da
  CO, le disfunzioni endocrine, le malattie sistemiche, l'ipossia, lo
  scompenso cardiaco e le malattie metaboliche derivanti da alterazioni
  renali o epatiche.
\end{itemize}

\subsection{Manifestazioni cliniche}

Per quanto riguarda le manifestazioni cliniche, il delirium si
caratterizza per \emph{obnubilamento ed alterazioni dello stato di
coscienza}, \emph{stato crepuscolare} e \emph{stato confusionale}.
L'esordio può essere rapido e brusco, nel giro di poche ore, oppure
anche dopo alcuni giorni di \emph{ansia}, \emph{insonnia} o
\emph{sonnolenza}, e la fase di stato si caratterizza per la presenza di
\textbf{allucinazioni transitorie}, spesso di tipo di \emph{visivo} (ad
esempio le macro/microzoopsie del delirium tremens), associate ad
\emph{incubi} e \emph{lentezza motoria}. Fondamentale è
l'\emph{oscillazione dei livelli di vigilanza}, che possono essere di
due tipi: di \textbf{iperattività} o di \textbf{ipoattività}.
L'ipoattività si caratterizza per \emph{scarsa lucidità e confusione},
mentre l'iperattività si ha soprattutto quando si correla con
l'assunzione di sostanze. Importante è anche il \emph{disorientamento
con perdita di riconoscimento di persone familiari oppure di falsi
riconoscimenti}. Il linguaggio tende a divagare, è incoerente, prolisso
ed incongruo. A livello cognitivo ovviamente abbiamo \textbf{disturbi
della memoria}, cioè viene limitata la capacità di registrare, osservare
e richiamare i ricordi, le capacità attentive e l'abilità nel risolvere
problemi.

Caratteristiche sono poi le \textbf{percezioni alterate}, perché
l'individuo ha un'alterata percezione per incapacità a discriminare gli
stimoli, non è capace di integrare le percezioni attuali da quelle
passate o dai propri processi emotivi. Il paziente con delirium,
inoltre, è un \emph{paziente agitato}, \emph{distraibile}, che ad ogni
stimolazione tende ad agitarsi perché non riesce ad integrare le
percezioni in una singola esperienza comune. Anche le
\emph{allucinazioni} e le \emph{illusioni} sono frequenti, e l'umore
tende alla disforia, all'irritabilità ed alle paure immotivate, oppure
si può anche avere una condizioni di euforia infantile.

In questi pazienti, inoltre, sono tipiche le \emph{alterazioni del ritmo
sonno-veglia} e la presenza di \textbf{\emph{tremori ``a battito
d'ala''}} (flapping tremor) a livello degli arti superiori più eventuali
altri sintomi correlati con la patologia di base.

\subsection{Classificazione}
Può essere fatta in diversi modi e secondo diversi criteri:

\textbf{\emph{Stato di coscienza}: }

\begin{itemize}
\item
  DELIRIO LUCIDO: è quello che si ritrova generalmente nelle patologie
  psichiatriche. I pazienti (depressi, schizofrenici, paranoici\ldots{})
  delirano, ma mantengono uno stato di coscienza lucido e non alterato.
\item
  CONFUSO: è quello che si manifesta durante un'alterazione dello stato
  di coscienza. Generalmente sono legati ad una patologia organica.
\end{itemize}

Ad esempio il delirio febbrile: un paziente con febbre alta può avere
anche una sintomatologia delirante ed allucinatoria o illusionale. In
questo caso il delirio è secondario ad una profonda alterazione dello
stato di coscienza. Oppure, ancora più paradigmatici sono i deliri
notturni dei pazienti dementi, che vanno incontro ad una
destrutturazione dello stato di coscienza (che può essere deliroide,
confusionale, ecc\ldots{}).

\textbf{\emph{Struttura del delirio}: }

\begin{itemize}
\item
  PARANOICALE: per quanto riguarda la struttura è il più
  sistematizzato,ben strutturato, elaborato, con nessi associativi che
  reggono l'intero complesso delirante, tanto che il paziente può essere
  talmente convinto del suo delirio da interpretare qualsiasi cosa in
  funzione di questo. Come suggerisce il nome, questo tipo di delirio è
  quello che caratterizza la \emph{paranoia}. Il paziente costruisce
  un'impalcatura coesa che sostiene il delirio. Vi sono nessi logici, su
  base interpretativa, che lo spiegano. Generalmente si trova nei
  disturbi deliranti cronici, che non poggiano su una patologia, ma
  sulla personalità.

Ad esempio: \emph{il delirio di gelosia nel paziente che riporta i due
minuti di ritardo della moglie, calcola la benzina e sostiene quindi,
dopo aver costruito questa struttura complessa, il tradimento}.

\item
  PARANOIDE:che ha delle tematiche ricorrenti, ma non sono così
  elaborate, e la strutturazione è piuttosto povera. Generalmente è
  presente in pazienti con una destrutturazione più o meno profonda,
  come la schizofrenia (che, come diceva Bleuler, crea una scissione e
  devasta la vita dell'individuo): difficilmente il paziente sarà in
  grado di strutturare un delirio così ben schematizzato come quello del
  caso precedente.

Ad esempio: \emph{il paziente riferisce di essere seguito dagli alieni,
ma non sa dare una spiegazione, oppure lo dice in modo meno elaborato}.

\item
  PARAFRENICO: che risulta alimentato da fenomeni allucinatori molto
  marcati, come allucinazioni visive, uditive, olfattive o gustative,
  che sono quindi connesse con la tematica delirante e la rafforzano. Il
  delirio parafrenico è l'elemento che caratterizza appunto la
  \emph{parafrenia}.

  Quindi è spesso determinato da fenomeni allucinatori: le allucinazioni
  dei depressi hanno sempre le medesime tematiche dei deliri olotimici;
  cioè se sentono delle voci facilmente queste dicono ``\emph{Sei
  indegno}'', oppure ``\emph{Hai commesso degli errori terribili}'',
  ``\emph{Sei ridotto sul lastrico}'', ``\emph{Finirai il resto dei tuoi
  giorni in galera}''. A volte oltre che allucinazioni uditive hanno
  allucinazioni visive: ancora una volta con i medesimi contenuti:
  magari vedono diavoli o le fiamme eterne infernali.

Le allucinazioni olfattive sono molto più frequenti nei depressi che
negli schizofrenici, sempre con i medesimi contenuti: iniziano a sentire
odore di zolfo, o di putrefazione (legato al tema ipocondriaco).
\end{itemize}

(N.B.: I termini paranoicale e paranoide non hanno nulla a che vedere
con il contenuto; non si tratta, ad esempio, di deliri persecutori)

\textbf{\emph{Caratteri genetico-formali}: }

\begin{itemize}
\item
  PRIMARIO: è, per definizione, il delirio della schizofrenia. Non è
  conseguenza di altri disturbi psichiatrici, se non la schizofrenia:
  emerge direttamente dalla destrutturazione che essa porta. Venne
  chiamato così perché il delirio caratteristico della schizofrenia è
  incomprensibile sul piano psicologico: anche provando a metterci nei
  suoi panni, non riusciamo a colmare il vuoto che c'è tra la percezione
  ed il significato che il paziente fornisce.

\emph{Caso clinico:} \emph{il paziente che, entrando nella stanza, dice
che sulla sedia che vede è sceso Gesù}.

Si dicono primari perché dietro non c'è patologia; o meglio, ci sono
tali vissuti che non sono ancora sintomi, della Schizofrenia.

Possono essere di tre tipi:

\begin{itemize}
\item[1.]
  \emph{Percezione delirante}: è l'\textbf{unico} ad essere un sintomo
  di primo rango di Schneider, quindi è l'unico ad essere altamente
  predittivo di schizofrenia. Indica una percezione di per sé corretta,
  a cui viene attribuito un valore semantico abnorme. Come esempio si
  può considerare quello della \emph{visione della sedia}, citato sopra:
  ed è bene ricordare dalla scorsa lezione la diagnosi differenziale tra
  percezione ed interpretazione delirante.
\item[2.]
  \emph{Intuizione delirante}: non poggia su una percezione, ma avviene
  come un'illuminazione. Esempio: \emph{Il paziente improvvisamente si
  convince di essere Gesù}.

Può essere presente nella schizofrenia, ma è molto frequente anche in
altri disturbi, come quelli affettivi. Esempio: \emph{un paziente
bipolare che, quando inizia la fase maniacale, la prima cosa a cui va
incontro è sempre un'intuizione delirante di tipo megalomanico-mistico,
convinto di essere Gesù e di dover salvare il mondo.}

\item[3.]
  \emph{Rappresentazione delirante}: è un po' più simile alla prima, ma
  non è una percezione del mondo esterno che sostiene il delirio, bensì
  un ricordo vero (nemmeno un'allucinazione della memoria, ovvero un
  processo psicopatologico in cui il paziente crede di ricordare cose
  che non sono mai avvenute, trattandosi quindi di un ricordo falsato).
  Sono ricordi veri a cui vengono attribuiti significati abnormi, come
  nella percezione delirante. Esempio: \emph{un paziente che ricordava
  lo schiaffo della madre datogli quando aveva 5 anni, si convinse dopo
  tempo di non essere figlio dei suoi genitori, ma di essere figlio di
  una famiglia nobile, a cui addirittura andò a chiedere parte
  dell'eredità}. Il delirio genealogico nasce da una rappresentazione
  delirante, ovvero il ricordo dello schiaffetto della madre.
\end{itemize}

(N.B.: Intuizione delirante e rappresentazione delirante possono essere
deliri primari, quindi presenti nella schizofrenia, ma possono anche
essere presenti in altre patologie, ad esempio sono frequenti nei
disturbi affettivi, ed essere deliri secondari)

\item
  SECONDARI o DELIROIDI: sono comprensibili a partire dalla patologia
  alla base che sostiene il delirio.

Ad esempio:

\begin{itemize}
\item[1.]
  DELIRI OLOTIMICI: Sono disturbi secondari a disturbi affettivi, sia
  che si tratti di fasi depressive che maniacali.

  I deliri depressivi si riconoscono in genere abbastanza facilmente:
  hanno modalità formali tipiche ed il contenuto solitamente è sempre lo
  stesso, al di là delle epoche, dello stato sociale, della latitudine
  ecc... Essi infatti generalmente vertono su tre temi fondamentali:
  \textbf{colpa},\textbf{rovina}, \textbf{ipocondria}. Quando si ha uno
  di questi elementi bisogna sempre sospettare che vi sia una
  depressione alla base.

  \textbf{COLPA}: intesa come la preoccupazione per la salvezza della
  nostra anima. Non è sempre facile da immaginare.

  Esempio: \emph{un paziente era angosciato, perché sosteneva di vedere
  dei carabinieri in borghese sotto casa che lo controllavano ed erano
  pronti per arrestarlo; il paziente per altro tentò anche di uccidersi,
  perché ormai era preso da questo stato di assedio.} In base al
  contenuto, questo delirio si rubricherebbe inizialmente come
  persecutorio; in realtà, considerando la forma ed il vissuto che ci
  sta dietro, ci si accorge che è un delirio di colpa: \emph{il paziente
  infatti aggiunse che aveva venduto un camper su internet, chiedendo ed
  ottenendo un prezzo che riteneva egli stesso eccessivo e quindi
  credeva di dover andare in galera} delirio di colpa. È importante
  distinguerlo perché, se si trattasse questo caso con un antipsicotico
  fermandosi alla prima diagnosi, ci accorgeremmo che peggiorerebbe, in
  quanto la base del delirio è affettiva eservirebbe quindi un
  antidepressivo.

  \textbf{ROVINA}: intesa come il delirio di non avere più i mezzi di
  sostentamento economico, di non avere più possibilità di sostenere la
  famiglia, di non avere più nulla.

  \textbf{IPOCONDRIA}: ha a che vedere con la salute fisica, corporea.
  Esempio: \emph{paziente che dice di saper per certo di avere un tumore
  al cervello, in seguito ad un episodio di visione offuscata.
  Addirittura un paziente fu ricoverato perché iniziò a suonare a tutte
  le case del paese per salutare tutte le persone, in quanto sentiva che
  il giorno seguente sarebbe morto} delirio ipocondriaco e di rovina.

  Queste tre tematiche sono spesso interconnesse, addirittura spesso
  questi pazienti fanno costruzioni deliranti che contemplano tematiche
  di colpa, di rovina o di ipocondria interconnesse tra loro.

  Generalmente i deliri di colpa interessano azioni compiute dal
  paziente, la qual colpa si traduce nell'indegnità attuale, peraltro
  spesso riferita al corpo, che risulta marcio o malato. (``\emph{Sono
  una persona indegna'' -- ``Sono moralmente insulso e merito la
  morte}'').

  Nei casi gravi possono comparire anche deliri di negazione o
  nichilistici, che in realtà possono sempre essere ricondotti alle tre
  grandi tematiche descritte sopra. Esempio: \emph{paziente nega
  l'esistenza di familiari, di organi o parti corporee (``non ho più lo
  stomaco, mi è stato sottratto il cuore''). }

  Le forme estreme di depressione delirante o psicotica configurano la
  cosiddetta \textbf{sindrome di Cotard} o di \textbf{megalomania
  rovescia}: in essa tutte queste tematiche, soprattutto quelle di
  negazione o nichilistiche, sono particolarmente accentuate. Il
  paziente generalmente è incontrovertibilmente convinto di essere
  morto, nonostante capisca e sappia di sembrare vivo agli altri: questo
  fatto è vissuto come una condizione di dannazione eterna.

  Una variante di questo delirio è la \emph{forma inversa}: il paziente
  si ritiene l'unico vivo al mondo, in mezzo a tutti gli altri che sono
  morti. Sembrerebbe positivo, in realtà è sempre vissuto come
  dannazione: nel caso specifico dell'esempio, il \emph{paziente si
  sentiva solo in un universo in cui non esisteva più niente, l'unico
  rimasto immerso nella solitudine e attorniato da ombre e parvenze, che
  rappresentavano i simulacri di piante, animali, persone}.
\item[2.]
  DELIRI CARATTEROGENI: sono legati allo sviluppo di tratti peculiari di
  personalità. Per esempio, una personalità rigida, sospettosa, fanatica
  può sviluppare un delirio di persecuzione, che in questo caso è
  secondario ad una personalità premorbosa.
\item[3.]
  Deliroidi in pazienti con \emph{disturbi psicosensoriali abnormi,}
  come i sordastri, che sono conseguenza di questi deficit organici.
\end{itemize}

\end{itemize}

\paragraph{Stato d'animo predelirante (Wahnstimmung)}

Deriva da `Wahn'=delirio e `Stimmung'=umore.

È una forma pre-delirante che si manifesta nella schizofrenia prima che
ci sia l'esordio del delirio vero e proprio. Consente di capire meglio
cosa sia la schizofrenia e da dove origini il delirio; è tipica
dell'esordio per PROCESSO, non c'è generalmente nell'esordio per
sviluppo. In ogni caso, non è necessariamente presente.

È un vissuto di estrema angoscia, di profonda ansia da parte del
paziente, che ha una durata transitoria: non può durare più di qualche
ora, giorno, settimana. Spesso il paziente, in uno stato di perplessità,
formula una domanda caratteristica che ricorre molto frequentemente:
afferma che sia successo qualcosa di stravolgente, che non capisce e non
sa definire, con una portata cosmica, mondiale, che coinvolge tutti e
che, però, lo riguarda in prima persona.

Lo stato d'animo predelirante è quella fase di transizione, che
necessariamente dura poco tempo, in cui il paziente, prima di esordire,
perde i significati consueti delle cose, degli oggetti, della propria
vita psichica. Inizia a dissolversi la comune trama di significati entro
cui viviamo, si perde la familiarità, l'ovvietà della vita, collassa il
significato univoco delle cose: tutto può significare qualsiasi cosa e
nello stesso tempo non significa niente.

Caso clinico: \emph{il paziente, in questo stato, entra nella stanza ed
è angosciato perché vede il microfono. Sa cosa sia, perché non ha
problemi di percezione, ma non riesce più ad attribuirgli un significato
ovvio: mentre noi pensiamo che il microfono sia lì perché qualcuno deve
parlare con esso, il paziente inizia a chiedersi perché sia lì.
Successivamente volge l'attenzione alle proprietà fisiognomiche (perché
è nero, perché ha quella forma a freccia volta verso di me?). }

Termina, ovviamente, con la percezione delirante, cioè quando il
paziente torna finalmente ad attribuire un significato alle cose:
tuttavia questo non è più normale e legato al senso comune, ma abnorme.
In questo momento l'angoscia si placa, il paziente è tranquillo perché
riesce nuovamente a trovare un significato, però non torna più indietro
e tutto quello che poi accadrà sarà diretto ad alimentare questa visione
delirante.

Sono difficili da vedere, perché è raro che il paziente giunga
all'osservazione medica in questa fase essendo fasi transitorie e brevi,
mentre è più facile che vi arrivi quando già delira.

Caso clinico: \emph{un ragazzino di 14 anni, profondamente angosciato,
che fa notare una crepa su un palazzo di fronte fuori dalla finestra,
dicendo di avvertire che tutto il mondo stia cambiando, tutto abbia
qualcosa di malato e qualcosa di enorme stia per succedere} (in realtà
questa è già una fase più avanzata rispetto allo stato d'animo
predelirante, che si chiama `esperienza della fine del mondo').

Caso clinico: \emph{una paziente di 16 anni che tentò il suicidio
ingerendo tutti i farmaci trovati in casa, in seguito all'angoscia
insostenibile che provava per il pensiero che qualcosa stesse per
accadere}. Nell'arco di qualche settimana esordì con un delirio e quindi
iniziò la schizofrenia(si trattava di un pieno stato d'animo
predelirante).

Caso clinico: \emph{un ragazzo di 18 anni che viaggiava in macchina con
gli amici e avvertì che rumori e colori fuori dal finestrino
cambiassero, fossero enigmatici; questo senso di estraneità diventava,
di minuto in minuto, sempre più forte, più profondo, fino alla
sensazione che tutto il mondo respirasse all'unisono con il suo
respiro.} Da qui poi sviluppò una struttura delirante-allucinatoria
complessa e questa fase era prodromica all'esordio imminente.

Caso clinico: \emph{un paziente di 16 anni, seguito nelle settimane
precedenti all'esordio, nella fase in cui da un disturbo di personalità
schizotipico stava passando alla schizofrenia vera e propria. Non passò
da una fase di Wahnstimmung, ma di settimana in settimana si vedeva
cheperdesse i significati consueti delle cose. Era colui che soffriva di
geometrismo patologico, cioè era convinto che gli amici gli disegnassero
dei triangoli di cui dovesse calcolare l'area: quando esprimeva questa
attività però ancora andava a scuola, aveva amici e manteneva quindi un
ancoraggio con la vita reale, sebbene fosse già distaccato dal senso
comune. Era anche manierato (corretto nel modo di presentarsi). Poi,
improvvisamente, esordì a distanza di una settimana: il manierismo
scomparve e il paziente chiese al dottore se volesse un reattore
nucleare.}

Domanda: \emph{esiste la possibilità di prevenire l'esordio di un
delirio in questo stato? }

Sì, sebbene si sia già al limite dell'esordio e sia una situazione
talmente breve che è raro si possa agire in questa fase. Però, finché
non si arriva all'ingresso nella psicosi vera e propria, ovvero al
delirio, si può intervenire.

\subsection{Terapia}

La terapia del delirium è eziologica, e prevede \emph{reidratazione},
\emph{protezione antibiotica} ed eventualmente la \emph{sedazione}, per
cui per la farmacoterapia si usano essenzialmente le
\textbf{benzodiazepine} (lorazepam) e gli \textbf{antipsicotici}
(olanzapina, ziprasidone, aripiprazolo ed aloperidolo). Ultimo presidio
che si può usare in questi pazienti è poi la \textbf{\emph{TEC}}
(Terapia Elettro-Convulsivante), che oggi si usa solo per le forme gravi
e refrattarie, in assenza di cause organiche di delirium.

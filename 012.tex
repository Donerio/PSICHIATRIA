\documentclass[]{article}
\usepackage{lmodern}
\usepackage{amssymb,amsmath}
\usepackage{ifxetex,ifluatex}
\usepackage{fixltx2e} % provides \textsubscript
\ifnum 0\ifxetex 1\fi\ifluatex 1\fi=0 % if pdftex
  \usepackage[T1]{fontenc}
  \usepackage[utf8]{inputenc}
\else % if luatex or xelatex
  \ifxetex
    \usepackage{mathspec}
  \else
    \usepackage{fontspec}
  \fi
  \defaultfontfeatures{Ligatures=TeX,Scale=MatchLowercase}
\fi
% use upquote if available, for straight quotes in verbatim environments
\IfFileExists{upquote.sty}{\usepackage{upquote}}{}
% use microtype if available
\IfFileExists{microtype.sty}{%
\usepackage{microtype}
\UseMicrotypeSet[protrusion]{basicmath} % disable protrusion for tt fonts
}{}
\usepackage[unicode=true]{hyperref}
\hypersetup{
            pdfborder={0 0 0},
            breaklinks=true}
\urlstyle{same}  % don't use monospace font for urls
\IfFileExists{parskip.sty}{%
\usepackage{parskip}
}{% else
\setlength{\parindent}{0pt}
\setlength{\parskip}{6pt plus 2pt minus 1pt}
}
\setlength{\emergencystretch}{3em}  % prevent overfull lines
\providecommand{\tightlist}{%
  \setlength{\itemsep}{0pt}\setlength{\parskip}{0pt}}
\setcounter{secnumdepth}{0}
% Redefines (sub)paragraphs to behave more like sections
\ifx\paragraph\undefined\else
\let\oldparagraph\paragraph
\renewcommand{\paragraph}[1]{\oldparagraph{#1}\mbox{}}
\fi
\ifx\subparagraph\undefined\else
\let\oldsubparagraph\subparagraph
\renewcommand{\subparagraph}[1]{\oldsubparagraph{#1}\mbox{}}
\fi

% set default figure placement to htbp
\makeatletter
\def\fps@figure{htbp}
\makeatother


\date{}

\begin{document}

Isteria e Disturbo Algico

\emph{Isteria}

L'\textbf{isteria}, nota anche come \emph{disturbo di conversione} o
\emph{nevrosi isterica}, è una condizione caratterizzata dalla
\emph{presenza di uno o più sintomi neurologici}, come paralisi, cecità
o parestesie, che \emph{non possono essere spiegati da una malattia
neurologica o internistica nota}, ed in cui i sintomi d'esordio sono
tipicamente \emph{associati a fattori psicologici}.

La sindrome, oggi definita come disturbo di conversione, era in origine
associata alla sindrome ora nota come \emph{disturbo di somatizzazione}
e generalmente definita isteria, reazione di conversione o reazione
dissociativa. Attualmente si limita la diagnosi di disturbo di
conversione ai \emph{sintomi che coinvolgono una funzione volontaria
motoria o sensitiva}, cioè ai \textbf{sintomi neurologici}, e laddove il
medico non è in grado di spiegare i sintomi neurologici sulla base di
qualche malattia neurologica nota.

La diagnosi di disturbo di conversione richiede dunque che il medico
identifichi un'associazione necessaria e indispensabile tra la causa del
sintomo neurologico ed i fattori psicologici.

\textbf{Manifestazione Cliniche}

Clinicamente, il disturbo di conversione si manifesta con
\emph{paralisi}, \emph{cecità} e \emph{mutismo}, che sono i più comuni
sintomi di questa patologia, ma relativamente comuni sono anche i
\textbf{sintomi sensitivi}, come l'anestesia e le parestesie,
localizzate soprattutto agli arti. Tutti i sistemi sensitivi possono
essere coinvolti e la distribuzione delle alterazioni è solitamente
incongrua con quelle delle malattie neurologiche centrali o periferiche,
per cui si possono osservare le caratteristiche \emph{anestesie}
``\emph{a calza}'' o ``\emph{a guanto}'' oppure un'\emph{emianestesia
che inizia esattamente lungo la linea mediana}. I sintomi del disturbo
di conversione possono anche coinvolgere gli organi speciali di senso,
producendo \emph{sordità}, \emph{cecità} e \emph{visione} ``\emph{a
cannocchiale}'', e queste condizioni possono essere monolaterali o
bilaterali, tuttavia l'esame neurologico rivela che le vie sensoriali
sono integre, ad esempio, nella cecità da disturbo di conversione il
soggetto riesce a camminare senza riportare urti o danni, e le pupille
reagiscono alla luce, mentre i potenziali a livello corticale sono
normali.

Per quanto riguarda i \textbf{sintomi motori}, questi comprendono i
\emph{movimenti abnormi}, i \emph{disturbi della marcia}, al
\emph{debolezza} e la \emph{paralisi}. Possono essere presenti anche dei
\emph{tremori grossolani} e \emph{ritmici}, con \emph{movimenti
coreiformi}, \emph{tic} e \emph{sobbalzi}. I movimenti generalmente
peggiorano quando l'attenzione è focalizzata su di essi. Un peculiare
tipo di disturbo della marcia osservato nel disturbo di conversione è
l'\textbf{\emph{astasia-abasia}}, che è una \emph{deambulazione
vistosamente atassica}, \emph{vacillante}, accompagnata da
\emph{movimenti del tronco grossolani}, \emph{irregolari} e \emph{a
scatti} e da movimenti delle braccia a scatti e ondeggianti. I pazienti
con questi sintomi raramente cadono, e se questo succede, in genere non
si fanno male. Altri disturbi motori comuni sono le \emph{paralisi} e le
\emph{paresi}, che possono interessare uno, due o tutti e quattro gli
arti, anche se la distribuzione dei muscoli coinvolti non rispecchia
quella delle vie neurologiche. I riflessi sono normali, e non ci sono
fascicolazioni o atrofia muscolare, ed anche l'EMG è normale.

Si possono avere anche dei \textbf{sintomi simil-epilettici}, cioè delle
\emph{pseudocrisi epilettiche} che sono difficili da distinguere da
quelle reali, anche perché circa un terzo dei soggetti con pseudo crisi
ha anche un concomitante disturbo epilettico.

Infine, una particolare manifestazione clinica dell'isteria è il
fenomeno noto come ``\textbf{\emph{La Belle Indifférence}}'', cioè
l'\emph{atteggiamento del paziente inappropriatamente indifferente nei
confronti di un sintomo grave}, cioè il paziente non sembra preoccupato
da quello che appare invece come un deficit importante.

La diagnosi differenziale di disturbo di conversione non è affatto
semplice, poiché è alquanto complesso escludere in modo definitivo una
malattia organica, per cui la diagnosi differenziale va posta con
diversi \textbf{disturbi neurologici}, coi tumori cerebrali e le
patologie che interessano i gangli della base, nonché con
l'\textbf{ipocondria}, in cui il paziente non ha una vera e propria
perdita o alterazione del funzionamento, ed i disturbi somatici non sono
limitati ai sintomi neurologici, mentre si hanno dei caratteristici
atteggiamenti e forte convinzione della propria patologia.

\textbf{Trattamento dell'Isteria}

La risoluzione dei sintomi del disturbo di conversione è di solito
spontanea, anche se probabilmente è facilitata da una \emph{terapia di
supporto introspettiva}, per cui è fondamentale stabilire una relazione
terapeutica nella quale il terapista guidi con autorità ed assista il
paziente, per cui spesso si ricorre alla psicoterapia psicodinamica o a
forme brevi di psicoterapia.

\emph{Disturbo Algico}

Il \textbf{disturbo algico} (noto anche come \emph{disturbo da dolore
somatoforme} o \emph{disturbo da dolore psicogeno}), è definito dalla
presenza di \textbf{\emph{dolore}} che è ``l'oggetto principale
dell'attenzione clinica'', per cui il sintomo principale è proprio il
dolore, in uno o più distretti anatomici, che non viene completamente
spiegato da un'affezione medica non psichiatrica o neurologica.

I sintomi algici sono peraltro associati a \emph{disagio emotivo e a
limitazione del funzionamento del paziente}, ed il disturbo ha una
\emph{plausibile relazione causale con fattori psicologici}; questa
forma di disturbo, peraltro, è due volte più comune nelle donne che
negli uomini, e l'età di esordio più frequente è tra la IV e V decade di
vita, forse perché la tolleranza al dolore si riduce con l'età.

\textbf{Manifestazioni Cliniche}

I pazienti con disturbo algico non costituiscono un gruppo uniforme dal
punto di vista clinico, ma piuttosto sono un gruppo eterogeneo con vari
dolori, come \emph{lombalgia}, \emph{cefalea}, \emph{dolore facciale
atipico} e \emph{dolore} \emph{cronico pelvico}. Questi dolori possono
essere post-traumatici, neuropatici, neurologici, iatrogeni o
muscolo-scheletrici, tuttavia, per soddisfare la diagnosi di disturbo
algico occorre che sia presente un fattore psicologico che possa essere
considerato significativamente correlato ai sintomi algici e alle loro
ramificazioni. I pazienti con questo disturbo hanno spesso una
\emph{lunga storia di cure mediche e chirurgiche}, con consulti presso
molti medici e \emph{richiesta di numerose terapie}, e in alcuni casi
possono diventare anche piuttosto insistenti, arrivando a richiedere gli
interventi, e sono \emph{completamente coinvolti dalla preoccupazione
per il dolore}, riferendolo come \textbf{fonte di ogni loro disagio e
sofferenza}.

Inoltre, i pazienti con disturbo algico possono avere un quadro clinico
complicato da un disturbo correlato a sostanze, perché \emph{tentano di
ridurre il loro dolore con alcol o droghe}, mentre tipici sono anche
alcuni disturbi dell'umore, come la distimia o gli episodi depressivi
maggiori, che in questi casi possono presentarsi nella forma definita di
``\textbf{depressione mascherata}'' della psicopatologia classica.

\end{document}

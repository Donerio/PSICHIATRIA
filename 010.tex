\section{Disturbi deliranti cronici}

\subsubsection{Definizione}

I disturbi psicotici non affettivi non schizofrenici si dividono in due
grandi gruppi: i disturbi deliranti cronici e le psicosi acute brevi.

Paranoia e parafrenia, nei sistemi nosografici attuali, appartengono ai
disturbi deliranti cronici.
\\\\
Sono disturbi esclusivamente o prevalentemente caratterizzati dal
sintomo delirio, con tendenza alla cronicizzazione. Il delirio non
scompare e, se non trattato, tende ad ingrandirsi sempre più, con una
tendenza centrifuga.

Caso clinico: \emph{un delirio persecutorio inizialmente può interessare
la cerchia della famiglia del paziente (padre e madre che perseguitano
il paziente), per poi estendersi al paese in cui vive, poi all'Italia,
poi al mondo}.

Considerando la classificazione riportata sopra, in genere rientrano
tutti tra i deliroidi secondari, in cui il delirio insorge gradualmente
a partire da una personalità premorbosa. Sono degli sviluppi (sviluppo=
esordio subdolo e graduale, nell'arco di mesi o di anni) deliranti di
personalità.

\subsection{Paranoia}

È caratterizzata, in primo luogo, dal fatto che l'unico sintomo presente
è il delirio: non ci sono allucinazioni, disturbi del pensiero,
disorganizzazione ideativa, disturbi dell'umore.

Secondo la definizione di Kraepelin, è una \emph{malattia autonoma (sia
rispetto alla schizofrenia che ai disturbi dell'umore) e specifica,
caratterizzata dall'insidioso (lento, graduale) sviluppo (a partire da
personalità premorbosa) di un sistema delirante permanente e
incrollabile. }

Il delirio, una volta che si instaura, si espande ed arriva a devastare
la vita dell'individuo. Dal punto di vista della struttura, sempre
\textbf{paranoicale}: è un delirio ben sistematizzato ed elaborato. Il
paziente mette a disposizione tutte le sue capacità di memoria, di
intelligenza, cognitive etc\ldots{} per costruire il suo delirio. Non si
hanno altre manifestazioni psicopatologiche, non si ha deterioramento
delle funzioni mentali del paziente,né segni di destrutturazione della
personalità, come invece avviene nella schizofrenia. Insorge a partire
dai 30 anni.

I deliri della paranoia sono deliri con una \textbf{tendenza
centripeta}, e tendono ad evolversi sempre di più, assumendo comunque le
caratteristiche di un delirio lucido, senza allucinazioni, ma con
presenza di nessi logici, socialità integra, cura di sé mantenuta, ma
con un esistenza completamente divorata dal delirio. Citando un autore,
si può sia che si arrivi ad ``\emph{un'esistenza divorata dal
delirio''}: i pazienti finiscono per distruggere la propria vita, perché
tutto ruota attorno al delirio, che rimane l'unica ragione.
L'affettività è integra, la memoria è salda, la coscienza è lucida, non
si hanno disturbi psicosensoriali.

Dal punto di vista formale sono tutti sostenuti da una modalità
\emph{interpretativa}: si tratta di un lavorio continuo, che dura mesi,
basato su interpretazioni di dati della realtà, volti a confermare il
sospetto delirante. Esempio: \emph{paziente con delirio di gelosia, con
la moglie che rientrava a casa due minuti più tardi del solito.}
\\\\
Per quanto riguarda i contenuti, quelli classici più frequenti sono
deliri:

\begin{itemize}
\item
  \textbf{Erotomanici}: convinzione delirante di essere oggetto di
  interessi amorosi da parte di una persona, generalmente più bella, più
  elevata come rango sociale.

Il decorso naturale, in era pre-farmacologica, si caratterizzava per la
presenza di tre fasi: in un primo momento il/la paziente si convinceva
dell'idea ed era restio ad accondiscendere a questa proposta d'amore per
vari motivi, legati, ad esempio, al matrimonio (oggetto del delirio
cerca il paziente); nella seconda fase questo amore diventava
indissolubile e voluto, ma non riusciva a concretizzarsi per la presenza
di altre persone che lo osteggiavano (oggetto si unirebbe al paziente,
ma vi sono ostacoli); nell'ultimo momento il paziente arrivava a
sentirsi sedotto ed abbandonato dall'oggetto d'amore, reo di questo
inganno, e si concludeva con omicidio dell'oggetto/suicidio del
paziente.

Caso clinico: \emph{una paziente di 45 anni ricoverata in TSO con
delirio erotomanico in seconda fase, non seguita prima. Aveva visto
l'oggetto d'amore una volta nella sua vita in una palestra, non ne
conosceva il nome (anzi, lei stessa gliene aveva attribuito uno) e non
gli aveva mai parlato. Tutto il delirio era basato su modalità
interpretative: era convinta che, durante lo sguardo che si erano
scambiati, avesse fatto il gesto di sfilarsi l'anello, facendole una
promessa. Era convinta che tutto il mondo osteggiasse questo amore,
anche i medici che l'avevano ricoverata. Al momento del ricovero aveva
già divorziato, non vedeva più i figli, non lavorava più: era
completamente divorata dal delirio.}

Domanda: \emph{Nel delirio erotomanico il paziente diventa pericoloso
per l'oggetto del delirio solo in fase 3 o anche prima?}

Generalmente il delirio erotomanico non si associa ad episodi di
stalking, per esempio, che invece attinge ad altri elementi
psicopatologici. Il paziente non ha rapporti con l'oggetto d'amore, si
bea di questa relazione esclusiva, si crea un mondo distaccato senza
vedere edandare attivamente a cercare l'oggetto d'amore.

\begin{itemize}
\item
  \textbf{Interpretativo}: Il delirio interpretativo è la modalità
  formale tipica della paranoia, che genera un delirio sistematizzato,
  paranoicale. L'interpretatività ne è, anche se non sempre, ovviamente,
  l'elemento caratteristico. Quindi, delirio interpretativo si può
  intenderepraticamente come sinonimo di paranoia
\item
  \textbf{Di persecuzione}
\item
  \textbf{Di querela}
\item
  \textbf{Di grandezza}
\end{itemize}
\end{itemize}

Nella schizofrenia spesso i deliri sono fantastici e bizzarri
(coinvolgono altri mondi, alieni, servizi segreti\ldots{}), per cui
vengono riconosciuti da chiunque.

Nella paranoia invece, soprattutto nelle fasi iniziali, sono molto
legati alla realtà e possono addirittura avere un seguito, finché non
raggiungono dimensioni esagerate: coinvolgono l'amore, il lavoro, la
promozione, il fatto di sentirsi truffati, ecc\ldots{} In passato,
persone con paranoia hanno infiammato le folle, sono riuscite a creare
gruppi e sette, proprio perché inizialmente sono verosimili. Inoltre,
avendo i pazienti una personalità premorbosa, senza destrutturazione
della personalità, sono credibili e per molto tempo riescono a mantenere
una vita sociale normale. Per questo giungono tardivamente
all'osservazione del medico: il problema è che più tardivamente vi
arrivano ed iniziano la terapia farmacologica, più è difficile che il
delirio receda, poiché la cronicizzazione è insita in questi disturbi.
Devono essere colti e trattati tempestivamente, perché altrimenti non
possono essere sradicati: nell'arco del tempo cambiano l'intero assetto
cognitivo del paziente e la sua vita psichica, in funzione del delirio,
che diventa la ragione centrale senza possibilità di toglierlo.

Caso clinico: \emph{una persona che tutti i giorni stava in piazza
davanti ad una banca nelle ore di punta, sopra a delle pertiche, con un
megafono, urlando le nefandezze che la banca gli aveva causato} delirio
di persecuzione e ``querulomane''. Era chiaramente una paranoia, però
inizialmente poteva mostrare credibilità.
\\\\
Per fare diagnosi di paranoia, \emph{il delirio deve essere presente da
almeno un mese}, e si deve sempre ricordare che la paranoia \emph{emerge
sempre da una \textbf{personalità premorbosa}} (spesso si tratta di
persone rigide, con DP narcisistico, personalità sospettosa o
diffidente). L'esordio della paranoia è \emph{tra i 30 ed i 40 anni},
quindi più tardivo che nella schizofrenia.

I tratti di personalità premorbosa che predispongono alla paranoia
possono essere diversi, non c'è un unico tipo di disturbo.

Un esempio classico è il disturbo paranoide di personalità: chi per
natura è diffidente, sospettoso, è più facile che tenda a sviluppare una
paranoia.

Oppure, tratti narcisistici di personalità: caso clinico: \emph{un
paziente che da anni aspetta una promozione in ambito lavorativo, con
tratti narcisistici che lo sostengono, può sviluppare un delirio
persecutorio in seguito ad un evento trigger, come il conferimento della
stessa promozione ad un collega. }

Vi è sempre un'interazione tra una personalità premorbosa ed uno o più
eventi di vita, a volte insignificanti, ma che vanno a toccare degli
elementi di vulnerabilità del paziente tali da mettere in moto il
delirio. Gli eventi sono diversi, a seconda della natura della
personalità che vi è dietro, e vanno a creare dei nuclei nascosti
affettivi, di vergogna, di insoddisfazione, di insicurezza, che vengono
celati e coperti dal delirio. Esso diventa la `reazione granulomatosa'
che copre il `corpo estraneo', rappresentato dal nucleo affettivo.

Caso clinico: \emph{Il paziente di prima, che riceve la ferita
narcisistica, manifesta questo vissuto di vergogna ed inadeguatezza che
copre col delirio.}
\\\\
Si vedono in maniera evidente inoltre nelle personalità
insicuro-sensitive (che caratterizzano il \textbf{delirio di rapporto
sensitivo}, descritto per la prima volta da Kretschmer): in queste
persone coesistono sia un elemento di insicurezza, scarsa autostima,
vissuti profondi di inadeguatezza, che elementi sensitivi, stenici,
narcisisti. Sono elementi contropolari che possono favorire il delirio.
L'elemento di vergogna viene nascosto dall'elemento stenico, che è
quello che fa strutturare il delirio.

Esempio, \emph{delirio dei masturbatori: fu un esempio dello stesso
Kretschmer. Bisogna considerare la Germania di inizio Novecento e la
colpa che poteva essere correlata alla masturbazione. I ragazzini,
associati a questo vissuto profondo di vergogna, iniziavano a delirare
uscendo di casa e accorgendosi dello sguardo e dello scherno degli altri
per quello che avevano fatto.}
\\\\
Ne esistono anche delle forme particolari, come la ``\textbf{follia a
due}'', che si ha quando due o più persone delirano e si influenzano a
vicenda nel delirio: in questo caso abbiamo una personalità più forte,
che idealizza ed induce il delirio, ed una personalità più debole, che
invece ``accoglie'' il delirio.


\subsection{Parafrenia}

È la seconda forma di disturbo delirante cronico e ha una struttura
\textbf{parafrenica}, non paranoica. È un delirio cronico, ma non è
verosimile, né calato nella realtà quotidiana: ha caratteristiche di
contenuto simil-schizofrenico (immaginifico, fantastico, megalomanico,
cosmico).

È sostenuto ed alimentato da fenomeni psicosensoriali-allucinatori
floridi, generalmente polisensoriali: uditivi, visivi, olfattivi,
gustativi. L'età di insorgenza è uguale a quella della paranoia, quindi
età media o pre-senile, più tardiva rispetto all'età della schizofrenia
(16-25 anni, al massimo fino ai 30 anni).

Anche qui non vi è destrutturazione della personalità e non si ha
perdita dello stato di coscienza.

Si ha inoltre \emph{integrità del rapporto con la realtà} (doppio
binario): sono deliri immaginifici che però, a differenza della
schizofrenia, avvengono in pazienti che mantengono una vita
assolutamente normale e irreprensibile ed uno stretto rapporto con la
realtà. Infatti spesso giungono all'osservazione medica per caso,
passano in genere inosservati nonostante delirino per anni e possono
continuare per lungo tempo, a differenza dei paranoici.

Esempio: \emph{donna di 30 anni parafrenica, medico, che fino al giorno
prima del ricovero aveva operato. Aveva deliri misti persecutori,
erotomanici e mistici, immaginava angeli e inglobava persone della vita
reale. }
\\\\
Domanda: \emph{Come fa un parafrenico ad essere scoperto?}

In quest'ultimo esempio,\emph{il marito scoprì che teneva decine di
diari in un ripostiglio in cui annotava tutto il suo mondo delirante,
scritti per anni (aveva una graforrea).} Quindi, sostanzialmente,
vengono scoperti casualmente.
\\\\
Domanda: \emph{riguardo alla paranoia e alla parafrenia, abbiamo detto
che insorgono su una personalità pre-morbosa, questa personalità viene
mantenuta?}

La personalità non viene annullata dal disturbo ma è come se tratti già
preesistenti nella personalità diventassero ipertrofici; c'è una
transizione graduale dalla personalità al delirio vero e proprio senza
destrutturazione come nella schizofrenia.

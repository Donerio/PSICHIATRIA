\documentclass[]{article}
\usepackage{lmodern}
\usepackage{amssymb,amsmath}
\usepackage{ifxetex,ifluatex}
\usepackage{fixltx2e} % provides \textsubscript
\ifnum 0\ifxetex 1\fi\ifluatex 1\fi=0 % if pdftex
  \usepackage[T1]{fontenc}
  \usepackage[utf8]{inputenc}
\else % if luatex or xelatex
  \ifxetex
    \usepackage{mathspec}
  \else
    \usepackage{fontspec}
  \fi
  \defaultfontfeatures{Ligatures=TeX,Scale=MatchLowercase}
\fi
% use upquote if available, for straight quotes in verbatim environments
\IfFileExists{upquote.sty}{\usepackage{upquote}}{}
% use microtype if available
\IfFileExists{microtype.sty}{%
\usepackage{microtype}
\UseMicrotypeSet[protrusion]{basicmath} % disable protrusion for tt fonts
}{}
\usepackage[unicode=true]{hyperref}
\hypersetup{
            pdfborder={0 0 0},
            breaklinks=true}
\urlstyle{same}  % don't use monospace font for urls
\IfFileExists{parskip.sty}{%
\usepackage{parskip}
}{% else
\setlength{\parindent}{0pt}
\setlength{\parskip}{6pt plus 2pt minus 1pt}
}
\setlength{\emergencystretch}{3em}  % prevent overfull lines
\providecommand{\tightlist}{%
  \setlength{\itemsep}{0pt}\setlength{\parskip}{0pt}}
\setcounter{secnumdepth}{0}
% Redefines (sub)paragraphs to behave more like sections
\ifx\paragraph\undefined\else
\let\oldparagraph\paragraph
\renewcommand{\paragraph}[1]{\oldparagraph{#1}\mbox{}}
\fi
\ifx\subparagraph\undefined\else
\let\oldsubparagraph\subparagraph
\renewcommand{\subparagraph}[1]{\oldsubparagraph{#1}\mbox{}}
\fi

% set default figure placement to htbp
\makeatletter
\def\fps@figure{htbp}
\makeatother


\date{}

\begin{document}

\textbf{\emph{Disturbi del Comportamento Alimentare:}}

\textbf{\emph{DEFINIZIONE E CARATTERISTICHE GENERALI:}}

I \textbf{disturbi del comportamento alimentare} (\textbf{DCA}), secondo
le ultime classificazione del DSM-V, sono un gruppo molto ampio ed
eterogeneo di disturbi psichiatrici, che include al suo interno alcune
forme principali, nello specifico:

\begin{itemize}
\item
  l'\emph{anoressia nervosa} (AN)
\item
  la \emph{bulimia nervosa} (BN),
\item
  ma anche altre forme psichiatriche come il \emph{BED (Binge Eating
  Disease)}, il \emph{picacismo}, il \emph{disturbo di ruminazione} ed
  altre forme generalmente indicate come \emph{OSFED} (Other Specified
  Feeding or Eating Disorders).
\end{itemize}

Per definizione, inoltre, non si può parlare propriamente di DCA se non
sono rispettati i seguenti 4 criteri:

\begin{enumerate}
\def\labelenumi{\arabic{enumi}.}
\item
  Dev'essere presente un \textbf{disturbo o un'alterazione delle
  abitudini alimentari o dei comportamenti specificamente volti al
  controllo del peso}. Cioè come e quanto un soggetto mangia e quindi
  quanto e come un soggetto controlla il proprio peso.
\item
  Dev'essere presente nel paziente \textbf{\emph{un'eccessiva influenza
  della forma e del peso corporeo sui livelli di autostima}} (questo
  punto è quello fondamentale per comprendere i meccanismi alla base dei
  DCA, e prende il nome di \textbf{disturbo dell'immagine corporea}).
  Tutte le altre cose che generalmente sostengono l'autostima di una
  perosna sana per il paziente affetto da DCA perdono di significato. Se
  non c'è questo criterio non si può parlare di disturbo psichiatrico
\item
  I precedenti due punti devono causare un \textbf{deterioramento
  clinicamente significativo della salute fisica o del funzionamento
  psico-sociale del paziente} (come in molti altri casi psichiatrici, un
  certo atteggiamento o convinzione non può essere considerata
  patologica se non causa un cambiamento significativo del funzionamento
  del soggetto).
\item
  Dev'essere \textbf{esclusa la presenza di altre eventuali patologie
  internistiche o psichiatriche alla base}.
\end{enumerate}

Tra questi quattro punti, quello su cui si deve focalizzare l'attenzione
è il secondo, cioè il disturbo dell'immagine corporea, in quanto
costituisce la caratteristica psico-patologica principale dei DCA.

\textbf{\emph{CLASSIFICAZIONE:}}

I DCA vengono generalmente suddivisi in:

\begin{itemize}
\item
  \textbf{anoressia nervosa} (AN, con le due varianti AN-R e AN-P),
\item
  \textbf{bulimia} \textbf{nervosa}
\item
  i \textbf{DCA NAS} (Non Altrimenti Specificabili), che non rientrano
  né nell'AN, né nella BN.
\end{itemize}

\textbf{\emph{ETIMOLOGIA:}}

Se si va a guardare ai termini letterali delle patologie:

\begin{itemize}
\item
  ``anoressia'' starebbe a significare ``\emph{assenza di appetito}'',
\item
  ``bulimia'' vorrebbe dire ``\emph{fame da bue}'',
\end{itemize}

ma in realtà questi termini appaiono fuorvianti, perché in queste
patologie psichiatriche non si ha un'alterazione dell'appetito, sebbene
i pazienti arrivino spesso a negare il senso di fame, ma quello che
risulta anomalo è il disturbo dell'immagine corporea, in base al quale
l'autostima del paziente non può fare a meno di dipendere dalle sue
caratteristiche più strettamente fisiche, ovvero il peso e la forma del
corpo.

\textbf{\emph{EPIDEMIOLOGIA E MORTALITA':}}

Per quanto riguarda l'aspetto epidemiologico dei DCA, sono disturbi la
cui prevalenza nelle popolazioni a rischio, in particolare negli
adolescenti, sono in aumento:

PREVALENZA: per l'anoressia nervosa la prevalenza stimata è dello
\emph{0,5-1\%}, mentre per la bulimia nervosa la prevalenza varia
\emph{tra l'1 e l'8\%}.

ETA':Si tratta in ogni caso di disturbi che si sviluppano in genere
durante la \emph{seconda decade di vita,}

SESSO: Hanno una \emph{preferenza per il sesso femminile} (rapporto
maschi/femmine di circa 0,5 a 9,5 per l'anoressia nervosa, passa ad 1 a
4 per la bulimia)

TASSO DI GUARIGIONE: di questi pazienti meno del 50\% riesce a guarire
completamente

TASSO DI MORTALITA': il tasso di mortalità si attesta sul \emph{5-20\%},
cioè il valore più alto tra le varie patologie psichiatriche, assieme
all'abuso di sostanze, e questo perché nei DCA si ha un deterioramento
fisico notevolmente maggiore, a cui si va ad aggiungere anche il rischio
di suicidio, che tuttavia contribuisce solo per il 2\% a tutte le morti
dovute a DCA

\textbf{\emph{EZIOLOGIA:}}

Dal punto di vista dell'eziologia, l'origine dei DCA è complessa e
ancora non del tutto definita, sebbene vi siano due modelli principali
eziopatogenetici:

\begin{itemize}
\item
  \textbf{\emph{MODELLO BIO-PSICO-SOCIALE:}}

  Il primo è il \textbf{modello bio-psico-sociale}, in base al quale
  diversi fattori biologici, psicologici, familiari e socio-culturali si
  combinano in modo additivo per produrre una patologia del
  comportamento alimentare. Anche quando si era parlato dei disturbi
  della personalità avevamo detto che questo modello non ci soddisfa del
  tutto perché ci da un'idea del ``frullatore'', ossia frullando insieme
  questi fattori salta fuori la patologia. Abbiamo bisogno di un modello
  che ci spieghi qualcosa in più.
\item
  \textbf{\emph{MODELLO DIATESI STRESS:}}

  Il secondo modello, oggi più accreditato, è il \textbf{modello
  diatesi-stress}, in base al quale esiste una \emph{predisposizione
  biologica}, su base multigenica, che determina una
  \emph{predisposizione ai DCA e ne determina anche la forma} (ad
  esempio la vulnerabilità all'AN piuttosto che alla BN), ed è poi
  un'eventuale esposizione ad \emph{eventi stressanti} a determinare se
  il disturbo si svilupperà o meno. Quindi questo modello,
  vulnerabilità-stress, recita che una predisposizione biologica, ad
  esempio genetica, determina se io ho rischio di sviluppare un disturbo
  alimentare e in particolare se ho il rischio di sviluppare AN
  piuttosto che BN. La predisposizione da sola non fa il disturbo.
  L'esposizione a momenti di vita stressanti che possono essere sia
  psicologici (interni) che ambientali, determina a partire dalla
  predisposizione se il disturbo si svilupperà o meno. E' un po' la
  teoria gene-ambiente. Questo vuol dire che noi tutti differiamo nella
  probabilità di sviluppare un disturbo alimentare e dunque nella
  ``sensibilità'' agli eventi stressanti tipicamente associati
  all'eziologia di un dca.
\end{itemize}

Ovviamente, oltre ai fattori predisponenti su base genetica, ve ne sono
diversi altri di natura ambientale e culturale, Il fattore cardine che
dobbiamo spiegare nell'eziologia di questo disturbo è
\textbf{l'insoddisfazione per il corpo} che è una caratteristica comune
ad entrambe le patologia in essere.

Questa insoddisfazione per il corpo è dovuta anche a \textbf{FATTORI DI
TIPO CULTURALE} che inseriamo nei fattori di rischio di tipo ambientale
e sociale nei modelli precedentemente accennati.

In prevalenza questi disturbi hanno incidenza più alta in dei posti dove
c'è abbondanza di cibo, ma culturalmente viene \textbf{idealizzata la
magrezza} poiché essendoci rischio di obesità essa non è considerata
solo ``bella'' ma anche ``giusta'' perché è sana. Sana in quanto
l'obesità, l'ipercolesterolemia, il sovrappeso ecc... fanno morire un
sacco di gente da noi. I fattori di rischio culturali non sono solo
quelli passati in televisione (la modella magrissima che sfila), ma
anche il pediatra che a ragione ``prescrive'' al bambino in sovrappeso
una dieta, essendo tale patologia un fattore di rischio specie se
insorge in età pediatrica. Dunque queste culture danno alla magrezza non
solo un valore estetico, bensì anche un valore etico. Ci spiegano anche
la prevalenza in adolescenza perché è un momento in cui volente o
nolente il fisico cambia e il peso si fa sentire sotto forma di presa in
giro da parte dei coetanei o sotto forma di cambiamenti dovuti al
menarca. E' un età in cui insorgono anche cambiamenti dovuti all'enfasi
del soggetto sul corpo perché è importante per essere accettati dagli
altri, per trovare un fidanzati.

Altri fattori di rischio sono \textbf{l'influenza dei coetanei} (sono
state osservate delle piccole epidemie di anoressia in delle classi) e
\textbf{l'influenza dei media}. E' stato visto che questi fattori sono
tutti correlabili nell'insorgenza dei disturbi alimentari. Ce lo spiega
anche il fatto che l'incidenza di queste patologie nei paesi occidentali
è aumentata tra il 1930 e il 1980, con l'incremento dell'abbondanza e
della possibilità dunque di aumentare di peso

Quello che questi fattori culturali non ci spiegano è la stima
dell'\textbf{EREDITABILITÀ}: la proporzione di varianza fenotipica
ascrivibile a cause genetiche è alta. Se questi disturbi dipendessero
solo da fattori di rischio culturali, non ci aspetteremmo
l'ereditarietà. Questo vuol dire che la società ha colpa, ma c'è
qualcosa di più profondo sotto. Possiamo fare molto in termini di
prevenzione primaria, ma non tutto. Parliamo di cose che hanno a che
fare anche con la biologia, non sono solo dei capricci del soggetto che
segue i dettami della moda.

Dell'insoddisfazione per il corpo chi più e chi meno ne soffrono tutti
coloro i quali non abbiano una forma fisica invidiabile e questo è
sicuramente dovuto a fattori culturali, però essa da sola non fa un DCA.

Tra i fattori genetici, i principali sono i fattori che regolano il
\textbf{temperamento}, infatti si è visto che i DCA sono associati a
particolari tratti comportamentali (ad esempio, l'AN è tipica di
soggetti perfezionisti e dalla volontà forte, finanche ossessiva, mentre
la BN è più comune in soggetti impulsivi e con affettività negativa).

Infine abbiamo i \textbf{FATTORI FAMILIARI}, che sono in realtà alquanto
aspecifici, poiché presenti anche in molte altre patologie
psichiatriche; tra questi vanno ricordati la familiarità psichiatrica,
la storia di DCA materna, l'abuso fisico o sessuale e una storia
infantile di scarse cure parentali ma con elevato controllo
(\textbf{fenomeno dell'Over Protection}).

Per esempio: madre affetta da DCA non solo trasmette alla figlia la
predisposizione alla malattia, ma la sottopone in seguito a commenti
riguardo la sua forma e peso corporei.

Per esempio: L'ambiente familiare, quello del controllo senza affetto:
genitori iperprotettivi che fanno fatica ad accettare che i figli
facciano le cose in autonomia ma allo stesso tempo poco empatici, poco
capaci di dare affetto. In una percentuale non indifferente di casi c'è
abuso sessuale o fisico nell'infanzia.

\textbf{\emph{ESORDIO:}}

Ma come si sviluppa un disturbo dell'alimentazione? Alla base, come
accennato, vi sarebbe un \emph{personalità} \emph{predisponente},
caratterizzata da un \textbf{SENTIMENTO PERVASIVO DI INSICUREZZA E DA UN
SENSO DI SÉ INSTABILE}, che si accentua particolarmente all'inizio
dell'adolescenza; questi giovani pazienti sono spesso in leggero
sovrappeso, e vengono stimolati dai parenti, dai coetanei o addirittura
dal medico di medicina generale a perdere un po' di peso, essenzialmente
per prevenire ulteriori problemi fisici.

Quando questi fattori di rischio si uniscono in modo non casuale, per
via delle correlazioni gene-ambiente di cui prima, il soggetto si
affaccia all'adolescenza con una cosa che ha la caratteristica nucleare
di tutti i disturbi del comportamento alimentare, senza questa
caratteristica \textbf{l'insoddisfazione per il corpo} non si trasforma
in dca ed è indicata con il termine inglese ``unaffectiveness'' che
viene tradotto con l'italiano \textbf{\emph{\emph{``inadeguatezza}}''}.
E' un senso pervasivo di non sapere chi si è in ogni data circostanza e
non sapere come essere efficaci: come cambiare l'ambiente che hanno
intorno se non gli piace, come parlare con gli altri; queste persone vi
dicono ``\emph{mi sento non all'altezza}'', ``\emph{mi sento non
adeguato}'' davanti a tutto: scuola, genitori, amici.

Quindi c'è un \emph{problema di identità e di controllo del mondo
esterno}, non si sentono ``equipaggiati''. Ciò che per gli altri è
automatico e spontaneo per loro è impossibile. E' solo con questi
presupposti che arrivando all'adolescenza, incontrando i fattori
culturali che loro trasferiscono la ``soluzione'' al loro problema con
la ricerca estrema della magrezza: \emph{``se io riesco a controllare il
mio peso e ad essere magro, non avrò più questi problemi''}.

Questi pazienti affermano che prima non erano buoni a nulla ma ora poi
dimagrendo valgono. Questo è il punto cardine per capire i DCA e le
difficoltà di trattamento: \textbf{nessun peso è abbastanza basso}
(disturbo dell'immagine corporea) per i pazienti non è un problema ma
per loro è la soluzione. Di fatto conferisce senso e significato alla
vita del paziente e rappresenta la via finale dei fattori di rischio
osservato.

Il \textbf{disturbo dell'immagine corporea} comporta che tutta la
soddisfazione di sé dipende solo e soltanto dal peso corporeo che rende
la vita del paziente più semplice,più efficace e più sicura. Trova una
soluzione maladattativa alla sua sofferenza, confusione e senso di
inadeguatezza \emph{identificando tutto se stesso con il suo peso:}
valgo solo se sono sempre più magra. E' per questo che si tratta di una
patologia psichiatrica.

Trovandosi davanti ad un medico che sia psichiatra o internista che
sostiene che bisogna che il paziente riprenda peso per aver salva la
vita, il paziente diventa non collaborante perché sente che cercano di
sottrargli l'unica certezza della vita. E' una lotta incredibile.

Caso clinico: \emph{ragazzino di 13 anni con un BMI di 12.5 che
presentava iperattività (vedremo che è un meccanismo di compenso), calo
di peso repentino che con una frequenza di 34 bpm è scappato veloce come
il vento per i viali dell'ospedale tanta era la disperazione con cui si
opponeva al ricovero, una questione di vita o di morte per lui. I
genitori son dovuti corrergli dietro per recuperarlo, finché non è stato
predisposto il ricovero è stato letteralmente piantonato in casa con i
genitori ed i nonni perché in bagno di notte con la luce spenta saltava
nella doccia. Quando è arrivato in ospedale aveva già un versamento
pericardico e saltava ancora. Ridurre il più possibile il peso anche a
rischio di gravissime conseguenze per loro è davvero una questione di
vita o di morte.}

\textbf{\emph{DIAGNOSI:}}

Anche se non tutti i futuri medici faranno gli psichiatri potrebbero
ugualmente essere utili per riconoscere i prodromi di questi disturbi,
quindi per fare una diagnosi precoce si parte dall'\textbf{anamnesi}:

\begin{enumerate}
\def\labelenumi{\arabic{enumi}.}
\item
  \emph{Familiarità per disturbi psichiatrici} che per i soggetti che
  sviluppano AN in genere sono disturbi d'ansia o disturbo
  ossessivo-compulsivo, nel caso di BN si tratta invece di disturbi
  dell'umore, abuso di alcol o di altre sostanze.
\item
  Per entrambi c'è \emph{DCA materno} con \emph{basso peso alla nascita}
  perché la madre ha un DCA parzialmente guarito.
\item
  Ci possono essere \emph{complicanze neonatali} e \emph{problemi
  alimentari} nell'infanzia come rifiuto di alcuni cibi, alimentazione
  selettiva, disturbi gastroenterologici, ritardo nella crescita.
\end{enumerate}

In adolescenza, quelli che necessiterebbero di uno \textbf{screening}
sono:

\begin{itemize}
\item
  Soggetti a bassa crescita,basso peso e con disturbi mestruali. Il
  ginecologo per prima cosa prescrive la pillola ed è un grave errore,
  infatti va prima indagato se c'è un DCA.
\item
  Soggetti che si presentano con vomito e disturbi gastroenterologici
  non spiegati.
\item
  Pazienti affetti da diabete mellito di tipo I che presentano anche
  predisposizione per DCA (sono particolarmente a rischio perché si
  manifesta con perdita marcata del peso finché non si inizia il
  trattamento e il peso risale e si fa un regime dietetico controllato,
  di questo il paziente non sarà felice).
\item
  Soggetti che si presentano con disturbi depressivi od ossessivi (la
  psichiatria dell'adolescenza è più complicata, il paziente è il
  crescita!).
\end{itemize}

Se sarete medici di medicina generale, sarà importante per fare diagnosi
precoce saper riconoscere i segnali e approfondire le situazioni di
questi ragazzi e monitorare la crescita, dando molta attenzione alla
preoccupazione della gente che il paziente ha intorno.

La \emph{\emph{modalità di esordio}} più comune parte da una
\textbf{condizione di lieve sovrappeso} nell'adolescenza che innesca
l'inizio di una dieta e qui non siamo nella patologia, la dieta è giusta
e magari gli è stata data per una serie di ottimi motivi dal pediatra.

La situazione si \emph{\emph{sviluppa}} perché in questi soggetti ciò
che gli era stato prescritto come dieta dopo un po' lo dimenticano e
cominciano a ridurre sempre di più a partire dallo schema dietetico
l'apporto calorico.

Arrivano a mangiare sempre meno, il peso diminuisce molto e si assiste
alla \textbf{fase di ``luna di miele''} delle anoressiche a cui tutte
vorrebbero tornare: aumenta il tono dell'umore ed entrano in una
\textbf{fase ipomaniacale} (non è un episodio maniacale vero è proprio)
in cui la sensazione è quella di avere più forza, più energia, più
voglia di fare molte cose e si ha la sensazione che tutto venga
piuttosto facile. Per questo soggetto questa non è una sensazione usuale
e per la prima volta sente l'efficacia personale e interpersonale e non
chiederà di certo un trattamento.

Le cose progrediscono e si struttura il \textbf{disturbo dell'immagine
corporea}, : l'insicurezza e l'insoddisfazione del paziente vengono
placate dal controllo di quelle che sono le caratteristiche fisiche
immediatamente controllabili, cioè il peso e la forma del corpo:ossia il
paziente inizia ad avere la smania di controllare il suo peso e di
dimagrire sempre di più per potersi sentire adeguato. Per stare bene i
soggetti si chiudono in questa ``gabbia d'oro'' fatta solo del controllo
del peso. Il paziente giunge alla convinzione, spesso inconscia, che
controllando il proprio peso possa controllare tutti gli aspetti della
sua personalità che trova inadeguati, e ciò conferisce alla vita del
paziente un senso ed un significato che prima mancavano, in quanto
questi identifica tutta la sua persona nel peso e nella forma del corpo.
Qui siamo già entrati nella patologia.

Per mantenere il peso più basso possibile si strutturano le
\textbf{strategie per mantenere basso il peso corporeo},questi soggetti
fanno tutto quello che possono per ricercare la magrezza con i mezzi più
svariati:

\begin{enumerate}
\def\labelenumi{\arabic{enumi}.}
\item
  Il paziente cerca di perdere ulteriormente peso, in genere
  organizzando un'\emph{iperattività fisica pianificata e ritualistica}
  (cioè finisce per cercare il modo più faticoso di fare ogni cosa): è
  comune fare iperattività sistematica e ritualizzata ossia scelgo il
  modo più faticoso per fare tutte le cose (es: 8 piani di scale a piedi
  tutti i giorni invece di prendere l'ascensore).
\item
  Gli \emph{schemi alimentari iniziano a diventare rigidi}, fissi,
  stereotipati e di una povertà estrema e il paziente sceglie una serie
  di alimenti che considera accettabili per mantenere basso il peso. A
  questo punto manca qualsiasi tipo di spontaneità nelle scelte
  alimentari che sono dettate solo da questa necessità di controllo del
  peso e dunque dal fatto di accettare di mangiare solo ciò che non
  mette ansia legata al timore indicibile di ingrassare.
\end{enumerate}

Altra caratteristica tipica di questi pazienti è la \textbf{continua
negazione della fame}, la quale viene appunto definita come latente e
negata, e determina l'insorgenza di ossessioni sul cibo e sui regimi
alimentari, che sono ovviamente egodistonici.

Ovviamente se uno non mangia la prima cosa che gli viene è la fame, a
loro non manca l'appetito almeno non nelle prime fasi. Nelle ultime
invece vengono davvero compromessi i meccanismi di fame e di sazietà.

Loro negheranno questa fame anche se la sentono e rispunterà sempre
sotto forma di \textbf{ossessioni} (ideazioni, pensieri, immagini,
impulsi egodistoniche intrusive) sul cibo e sui regimi alimentari da
quando si alzano alla sera quando vanno a letto. A questa cosa ci si può
attaccare per convincerli a farsi curare perché è una situazione molto
disturbante per il soggetto.

Un altro modo che la fame ha di riemergere nella metà dei casi è sotto
forma di \textbf{crisi bulimica}. Questa lotta tra la volontà del
paziente ed il senso di fame può quindi evolvere in due diversi
disturbi:

\begin{itemize}
\item
  nel 50\% circa dei casi il senso di fame ha la prevalenza, spesso a
  causa anche del temperamento impulsivo del paziente, e si hanno
  \textbf{crisi bulimiche} con abbuffate di cibo e perdita di controllo,
  a cui fanno poi seguito degli spiccatissimi sensi di colpa che vengono
  ``espiati'' tramite il vomito autoindotto, l'abuso di diuretici e
  lassativi e l'uso di sostanze anoressizzanti, per cui questi pazienti
  soffrono di \textbf{\emph{bulimia nervosa}} o di
  \textbf{\emph{anoressia nervoso di tipo purging}} (la differenza
  risiede essenzialmente nella diversità di frequenza delle abbuffate),
\item
  mentre nel restante 50\% circa dei pazienti la volontà del paziente
  riesce a prevalere sul senso di fame, ed il paziente \emph{mantiene le
  restrizioni caloriche}, ma sempre senza saltare i pasti, ed il peso
  continua a scendere, e tale comportamento è caratteristico
  dell'\textbf{\emph{anoressia di tipo restrittivo}}.
\end{itemize}

La caratteristica che divide in due questa classe di soggetti è
l'abbuffata che caratterizza i soggetti che hanno disturbi bulimici e
invece i soggetti che hanno disturbi da anoressia nervosa restrittiva
non si abbufferanno mai.

L'esordio dal punto di vista sintomatologico è comune per tutti, ma ciò
che fa andare alcuni individui verso un disturbo ed altri verso l'altro
è la predisposizione genetica che abbiamo visto poco prima, ossia la
familiarità con patologie psichiatriche diverse.

La possibilità di sviluppare l'AN-R o la BN o la AN-P dipendono quindi,
come già accennato, dalle caratteristiche temperamentali del paziente:
in generale, i pazienti con caratteristiche genetiche di ipercontrollo,
ossessività e rigidità tendono a sviluppare l'anoressia restrittiva,
mentre chi ha delle caratteristiche di discontrollo impulsivo e
instabilità affettiva tendenzialmente seguirà la via dei disturbi di
tipo bulimico.

L'ambiente non determina la transizione da un disturbo all'altro ma è
importante istruire i genitori su quei comportamenti da non tenere con
questi ragazzi perché potrebbe spingerli a mangiare per senso di colpa
verso i genitori ond'evitare poi andare a vomitare perché ``non se lo
possono permettere''.

\textbf{\emph{CRITERI DIAGNOSTICI DELL'ANORESSIA NERVOSA}}

Per poter parlare propriamente di anoressia nervosa, devono essere
soddisfatti alcuni criteri dettati dal DSM:

\begin{itemize}
\item
  \textbf{Rifiuto di mantenere il peso corporeo al di sopra o in
  corrispondenza del minimo peso ideale per l'età e la statura del
  paziente}, che in genere corrisponde all'\textbf{85\%} del peso
  normale per quel paziente. Non c'è un cutoff preciso, ma già sotto 18
  di BMI in presenza degli altri sintomi si può parlare di sottopeso
  clinicamente significativo.
\item
  \textbf{Intensa paura di acquistare peso o di ingrassare anche se si è
  marcatamente sottopeso}.
\item
  \textbf{Alterazione del modo in cui il paziente vive il peso e la
  forma corporea (disturbo dell'immagine corporea), oppure un'eccessiva
  influenza del peso e della forma sui livelli di autostima o rifiuto di
  ammettere la gravità della situazione di sottopeso}.
\item
  Nel DSM-IV era inclusa come criterio diagnostico anche
  l'\textbf{amenorrea}, cioè l'assenza di almeno 3 cicli mestruali
  consecutivi, ma nel DSM-V questo criterio è stato rimosso, sia perché
  non lo si può ovviamente applicare ai soggetti di sesso maschile, sia
  perché nelle pazienti molto giovani può essere difficile identificare
  tale condizione.
\end{itemize}

\emph{\emph{Classificazione:}}

Ci sono due sottotipi di AN:

\begin{enumerate}
\def\labelenumi{\arabic{enumi}.}
\item
  l'\textbf{anoressia nervosa restrittiva} (\textbf{AN-R}) è tipica dei
  soggetti con ipercontrollo, che continuano le restrizioni caloriche
  nonostante il notevole senso di fame,
\item
  l'\textbf{anoressia nervosa di tipo purging} (\textbf{AN-P}): si hanno
  delle abbuffate o delle condotte di eliminazione, quali vomito
  autoindotto, abuso di lassativi, e questi atteggiamenti possono
  ricordare quelli della bulimia nervosa.
\end{enumerate}

\begin{quote}
DD CON BULIMIA NERVOSA da cui si distinguono per la \emph{frequenza
delle abbuffate} (sporadiche nell'AN-P, molto più comuni nella BN), il
\emph{peso} (nella BN il paziente è spesso normopeso o di poco
sottopeso, mentre nell'AN-P il paziente è marcatamente sottopeso), la
\emph{possibile assenza di amenorrea} (che comunque non è più un
criterio valido nel DSM-V) e per il fatto che nell'anoressia purging
\emph{le abbuffate possono non essere reali}, cioè il paziente cede alla
fame e mangia, magari poco, però poi si sente in colpa come se si fosse
abbuffato in maniera enorme, perché comunque è venuto meno al suo
obiettivo.
\end{quote}

\emph{\emph{Grado di Gravità:}}

Per il criterio di gravità in genere si usa il BMI:

\begin{enumerate}
\def\labelenumi{\arabic{enumi}.}
\item
  Leggera \textless{}17
\item
  Moderata tra 16 e 17
\item
  Grave tra 15 e 16
\item
  Estrema \textless{}15
\item
  Regime di emergenza: si fa riferimento ad un BMI di 14, anche se non
  si fa riferimento solo a quello.
\end{enumerate}

Ci possono essere forme sotto soglia per cui il soggetto può aver avuto
un periodo di AN che poi è andato in regressione grazie al trattamento e
può avere alcuni dei sintomi ma non tutti. In questo caso si parlerà di
remissione parziale.

E' importante anche capire che una persona che fa tutte queste cose
oltre al calo drammatico del peso ha anche altre ripercussioni fisiche e
psicologiche.

\emph{\emph{Ripercussioni psicologiche}}: \emph{sintomi depressivi,
ansiosi o ossessivi}, \emph{isolamento sociale} (il pensiero del
paziente è del tutto focalizzato sul controllo del peso, per cui si ha
un disinteresse verso gli altri e inoltre le condotte di controllo
sull'alimentazione o le abbuffate e il vomito autoindotto non permettono
una vita sociale normale: i pensieri fissi sul cibo non permettono di
fare serenamente le normali attività come andare al cinema e in più è
impossibile andare a prendere una pizza con gli amici, mangiare in
pubblico è un'impresa), \emph{tratti perfezionistici maladattativi}, con
uno \emph{stile cognitivo molto rigido}, \emph{disinteresse sessuale} e
\emph{pensiero magico}. sintomi depressivi, ansiosi e oppressivi,
saranno isolati socialmente).

Alcune caratteristiche cliniche sono esacerbate o causate dallo stato di
denutrizione e l'unico modo per curare l'anoressia è la renutrizione.
Qualsiasi farmaco prima della renutrizione è inutile o addirittura
dannoso. Anche una valutazione psicologica non è completa se prima non è
stato renutrito, ha un modo di pensare diverso dal suo modo usuale,
alterato perché c'è stato un cambio di personalità.

\emph{\emph{Ripercussioni fisiche}}: Clinicamente molto rilevanti sono
poi le \textbf{\emph{manifestazioni internistiche}}, che possono essere
estremamente gravi e potenzialmente letali. In corso di AN gli apparati
coinvolti sono praticamente tutti:

\begin{itemize}
\item
  \textbf{Apparato cardiovascolare}: E' un criterio valutato per
  l'ospedalizzazione; è comune l'\emph{ipotensione ortostatica}, spesso
  associata ad \emph{acrocianosi}, \emph{alterazioni dell'ECG},
  \emph{bradicardia}, \emph{allungamento del tratto QT a causa
  dell'ipopotassemia} e \emph{predisposizione alle aritmie} (sono la
  prima causa di morte). Per questo alcuni farmaci se il paziente non è
  stato renutrito sono potenzialmente dannosi perché gli antipsicotici
  per cui c'è lieve evidenza che aiutano nell'iperattività fisica per
  consumare calorie, aumentano il tratto QT, quindi capite il rischio.
\item
  \textbf{Apparato muscolo-scheletrico}: deplezione della massa
  muscolare e osteopenia che può progredire fino ad osteoporosi e può
  causare fratture. Questo succede perché non vengono sintetizzati gli
  estrogeni, cioè l'amenorrea nei soggetti di sesso femminile è dovuta
  al fatto che l'ipotalamo non manda dei messaggi per secernere in modo
  pulsatile, come sarebbe atteso dopo la pubertà, le gonadotropine e
  dunque gli ormoni ovarici che servono anche a fissare il calcio dal
  sangue alle ossa, dunque dopo sei mesi di amenorrea le linee guida
  dicono che bisogna fare la MOC (Mineralometria Ossea Compiuterizzata)
  anche se il soggetto ha 15 anni
\item
  \textbf{Tiroide}: si mette un po' a riposo, c'è la sindrome della
  tiroide che sta bene ma si ammala (euthyroid sick syndrome) e
  aumentano i livelli periferici di rT3 che dimostra che la tiroide
  funziona ma che l'ormone tiroideo viene inattivato in periferia poiché
  l'ormone tiroideo aumenta il metabolismo e il corpo si difende da
  questo poiché è l'ultima cosa di cui ha bisogno in questa situazione;
\item
  \textbf{Apparato gastrointestinale}: stipsi poiché mangiando
  pochissimo hanno anche poco materiale da espellere. Per i soggetti
  anoressici questa sensazione di gonfiore addominale è inaccettabile e
  questo può portare anche all'abuso dell'uso di lassativi;
\item
  \textbf{Pelle}: spesso cosparsa di peluria lunga e sottile chiamata
  lanugo, specie in fase avanzata del disturbo. Questo accade perché non
  sono più prodotti gli estrogeni e sono loro che con lo sviluppo dei
  caratteri sessuali secondari stimolano la crescita dei peli più in
  alcuni posti che in atri;
\item
  \textbf{Sistema Nervoso Centrale:} abulia, apatia, deterioramento
  cognitivo soprattutto in fasi croniche avanzate ed è questo il motivo
  per cui è quasi impossibile comunicare con loro prima di averli
  renutriti. \textbf{SNC} si riscontra \emph{deterioramento cognitivo},
  \emph{abulia}, \emph{apatia}, \emph{umore depresso e disforico}, ed
  \emph{idrocefalo ex-vacuo} per riduzione della sostanza cerebrale.
\item
  \textbf{Apparato riproduttivo}; si hanno \emph{amenorrea},
  \emph{arresto o regressione dello sviluppo sessuale},
  \emph{ipoestrogenemia} e \emph{pattern prepuberali di secrezione di
  FSH ed LH}
\item
  \textbf{Nel} \textbf{sistema emopoietico} si sviluppa \emph{anemia}
\item
  \textbf{Nel sistema endocrino-metabolico} si riscontra
  \emph{ipercortisolemia}, \emph{anomalie elettrolitiche} e
  \emph{sindrome da rT3} (reverse T3)
\item
  A \textbf{livello urinario} si riscontra \emph{iperazotemia},
  \emph{calo della GFR} e \emph{nefropatia ipovolemia}..
\end{itemize}

\textbf{\emph{DIAGNOSI DIFFERENZIALE:}}

La \textbf{\emph{diagnosi differenziale}} va posta con \emph{gravi
deperimenti secondari a malattie organiche}, \emph{disturbi endocrini o
patologie gastro-intestinali} (RCU, ulcera peptica, tumori,
enterocoliti), nonché altre patologie psichiatriche, quali la
\emph{depressione maggiore} (soprattutto la forma melanconica c'è
iporessia), la \emph{schizofrenia con delirio di veneficio} (si
convincono che gli altri vogliano avvelenarli) ed alcuni \emph{disturbi
fobici} con attacchi di panico situazionali con paura di ingoiare per
rischio di soffocare. Il fattore discriminante per la diagnosi
differenziale è la \textbf{presenza o meno del disturbo dell'immagine
corporea}, la quale non è mai un'ossessione, bensì un'\emph{idea
prevalente}, cioè sottesa da un fondo affettivo intenso e che ha un fine
ed è egosintonica (le ossessioni possono esserci, ma sono rivolte verso
il cibo e verso i regimi alimentari, e sono egodistoniche, cioè danno
fastidio al paziente).

\textbf{\emph{CRITERI DIAGNOSTICI DELLA BULIMIA NERVOSA}}

I criteri per la corretta definizione della bulimia nervosa, secondo il
DSM-V, sono i seguenti:

\begin{itemize}
\item
  \textbf{Presenza di ricorrenti abbuffate}, cioè \textbf{episodi di
  assunzione di ingenti quantità di cibo in un periodo di tempo ben
  definito} (di solito inferiore alle 2 ore, spesso di circa 30 minuti),
  durante il quale il paziente ha la \textbf{sensazione di perdere il
  controllo} e di non riuscire a fermarsi (\emph{senso di
  derealizzazione}).
\item
  \textbf{Presenza di ricorrenti ed inappropriate condotte compensatorie
  per prevenire l'aumento ponderale legato all'abbuffata}, come vomito
  auto-indotto, l'abuso di lassativi, di diuretici, il digiuno o
  l'esercizio fisico eccessivo.
\item
  \textbf{Le abbuffate e le condotte compensatorie si devono verificare
  in media almeno 2 volte a settimana per 3 mesi}.
\item
  \textbf{I livelli di autostima sono indebitamente influenzati dalla
  forma e dal peso corporei}.
\end{itemize}

Inoltre, si è visto che vi è un'\textbf{elevata comorbilità
psichiatrica} nei parenti di primo grado dei soggetti con BN, che
tendono a sviluppare anch'essi dei DCA, ma anche \emph{disturbi
dell'umore}, \emph{abuso di alcol e sostanze} ed \emph{obesità}.

\emph{\emph{Classificazione:}}

Anche all'interno della bulimia si possono poi osservare \emph{due
sottotipi distinti}:

\begin{itemize}
\item
  \textbf{\emph{BN con Condotte di Eliminazione}}, in cui il paziente
  mette in atto degli immediati meccanismi tesi ad eliminare l'effetto
  del cibo ingerito sull'aumento del peso.
\item
  \textbf{\emph{BN senza Condotte di Eliminazione}}, in cui i
  comportamenti compensatori inappropriati sono dati dal digiuno o
  dall'esercizio fisico eccessivo.
\end{itemize}

La bulimia nervosa è probabilmente la forma più tragica tra i disturbi
alimentari, infatti i soggetti con AN-R sono per certi versi dei
``vincitori'', nel senso che la loro volontà è sufficientemente forte da
resistere al senso di fame, per cui non hanno sensi di colpa, mentre le
pazienti con BN ad un certo punto falliscono ed \emph{incorrono in
\textbf{ricorrenti abbuffate}}, a seguito delle quali cadono vittime dei
\emph{sensi di colpa} e mettono in atto dei \emph{meccanismi
auto-punitivi}, come il vomito auto-indotto e l'uso smodato di lassativi
e diuretici Altro elemento utile per la distinzione è che
\emph{nell'AN-P le abbuffate sono soggettive}, cioè il paziente ha la
percezione di subire un'enorme perdita di controllo nei confronti del
cibo, anche se in realtà mangia comunque pochissimo.

Le AN-P, quindi, sono le pazienti con DCA più sfortunate, perché hanno
un'\emph{innata tendenza al perfezionismo e sono molto rigide
caratterialmente}, ma hanno anche una \emph{forte impulsività}, per cui
sono le pazienti col \emph{più alto rischio di suicidio}.

In realtà, la bulimia nervosa non è una singola entità patologica, per
cui sarebbe più opportuno parlare d\textbf{\emph{i sindromi bulimiche}},
le quali possono poi essere generalmente suddivise in \textbf{BN
semplice}, o \textbf{uni-impulsiva}, in cui la personalità è intatta e
si hanno spesso dei tratti perfezionistici e compulsivi, per cui si
tende a considerarla un po' meno grave, e la \textbf{BN borderline} o
\textbf{multi-impulsiva}, in cui si ha una disregolazione più ampia
dell'affettività e dell'autocontrollo, con spesso fenomeni associati di
autolesionismo, tentativi di suicidio, rabbia improvvisa ed immotivata,
abuso di sostanze, furto compulsivo ed attività sessuale promiscua.

Anche qui c'è il \textbf{disturbo dell'immagine corporea} con
conseguente influenza eccessiva sulla vita del soggetto di forma e peso
corporei e ciò non avviene unicamente durante un episodio di AN.

Può succedere che un soggetto che soddisfa questi tre criteri, ma anche
quelli dell'AN, la diagnosi almeno in quel momento verterà su AN
sottotipo con abbuffate e condotte di compenso.

\emph{\emph{Grado di gravità:}}

Gli indicatori di gravità in questo caso sono la frequenza e le condotte
compensatorie inadeguate, quindi si parla di:

\begin{enumerate}
\def\labelenumi{\arabic{enumi}.}
\item
  Lieve se c'è una media di 1-3 episodi in una settimana
\item
  Moderata 4-7 episodi in una settimana
\item
  Grave \textgreater{}7 episodi a settimana
\end{enumerate}

Le condotte di compenso sono spesso quelle che danno più problemi anche
dal punto di vista fisico. Anche in questo caso il soggetto può avere
una bulimia piena che poi va in remissione dopo il trattamento, può
stare meglio ma non del tutto e quindi è prevista la dicitura di BN in
remissione parziale.

\emph{\emph{Epidemiologia: }}

Epidemiologicamente, la bulimia nervosa è leggermente \emph{più
frequente rispetto all'anoressia nervosa}, avendo un prevalenza nella
popolazione generale che si attesta tra l'1 ed il 4\%, ma che sale a
valori di 3,8-8\% nelle popolazione a rischio, ciò i liceali e gli
studenti universitari. La bulimia nervosa ha un rapporto femmine/maschi
di 8 a 2, e l'esordio è leggermente più tardivo rispetto all'anoressia
nervosa, tra i 12 ed 35 anni, con un picco attorno ai 18 anni.

Le categorie a rischio sono le modelle, le ballerine e gli atleti di
ambo i sessi poiché per le performance richieste è importante mantenere
basso il peso corporeo.

Negli atleti per esempio un periodo di allenamento intenso può
precipitare l'esordio della BN. Nei familiari di primo grado figurano i
disturbi psichiatrici che avevamo già citato: dca, disturbi dell'umore e
obesità.

\emph{\emph{Esordio:}}

L'esordio è tipicamente quello che avevamo visto per AN, ossia un
periodo di regime dietetico per via di predisposizione all'obesità,
precedentemente c'è stato un episodio di AN di tipo restrittivo
associato alle caratteristiche ipomaniacali di esordio, ma dopo circa un
anno compaiono abbuffate cui segue il vomito autoindotto.

\emph{\emph{Crisi bulimiche:}}

A differenza di AN restrittiva, i soggetti che soffrono di BN hanno una
caratteristica importante cioè che le \textbf{abbuffate} sono percepite
come \textbf{egodistoniche}: il soggetto è costretto ad abbuffarsi
perché è una situazione che è completamente fuori dal suo controllo e
lui è disperato e quindi è per ridurre le abbuffate che chiede
trattamento. Il soggetto dunque non è consapevole del disturbo di
immagine per cui nessun peso è abbastanza basso, non chiede un
intervento per questo ma lo chiede perché volendo mantenere il peso più
basso possibile le abbuffate possono essere un ostacolo a questo fine.

Nella fase di malattia conclamata c'è un completo \textbf{sovvertimento
del regime alimentare}: il sogetto alterna crisi bulimiche a momenti di
digiuno completo che portano ad una marcata oscillazione del peso.

La condizione di questo soggetto lo porta ad \textbf{isolamento sociale}
più che l'AN perché questi soggetti sono costretti a fare proprio ciò
che non vogliono e dunque sono disperati tanto da compromettere
relazioni sociali, andamento scolastico ecc.., infatti l'abbuffata è un
evento che letteralmente distrugge la vita del paziente a causa del
disturbo dell'immagine corporea, da cui deriva la tragedia psicologica
di questi pazienti, che sono costretti ad abbuffarsi anche se non
vogliono, con conseguente \emph{crollo dell'autostima e marcato disgusto
per sé stessi}, poiché hanno fallito nel controllo del proprio peso, che
è l'unica cosa che conta per loro.

L'\textbf{abbuffata} è un episodio che dura meno di due ore, in genere
mezz'ora, è giornaliera ed è preceduta da stati d'animo spiacevoli che
però non vengono riconosciuti dal paziente perché c'è una specie di
cortocircuito emotivo.

A volte la vista di cibi proibiti scatena l'episodio bulimico che è
caratterizzato dall'ingente quantità di cibo ingerito che ha un elevato
contenuto calorico soprattutto in carboidrati e grassi, inoltre deve
avere una consistenza tale da poter essere ingurgitata voracemente. A
questa episodio è correlata una sensazione di perdita di controllo.

Il termine abbuffarsi psichiatrico non è paragonabile a quello che noi
usiamo per indicare un pasto eccessivo, è mangiare 100 cose diverse
senza far caso al sapore, può essere anche tutto ciò che io trovo in
frigo. Si mangiano delle cose che noi nemmeno riusciremmo a pensare.

L'ingestione di grandi quantità di cibo rappresenta un problema
``fisico'' in quanto dopo l'abbuffata alcuni soggetti fanno addirittura
fatica a respirare e a vomitare ed è per questo che ingeriscono oltre a
cibi solidi, anche la giusta quantità di liquidi quali yogurt e latte
per rendere facilitato il vomito dopo. La sensazione di ``ripienezza''
da una base ``concreta'' nella realtà alla visione della propria
immagine distorta e aumenta l'ansia dei pazienti.

Inoltre l'individuo non mangia perché ha desiderio di mangiare, il suo
desiderio è quello di essere sempre più magro. Ci sono dunque dei
fenomeni di derealizzazione, di estraneità perché non riesce a
controllare le sue azioni.

Per questo a tutto ciò fanno seguito sentimenti come colpa,
autodisprezzo, disgusto di sé e depressione...hanno fatto proprio la
cosa che non volevano. Sono delle sensazioni fortissime di disperazione
perché hanno fallito in quello che è lo scopo primario della loro vita.

Se ci sono questi sentimenti ne segue automaticamente la necessità di
espellere il cibo che si è ingerito nel modo che esso interferisca il
meno possibile con la necessità del soggetto di essere magro.

Questi pazienti quindi si inducono il vomito e addirittura nelle fasi
avanzate di malattia non hanno nemmeno più bisogno di stimolarlo con le
dita in gola ma hanno imparato a farlo automaticamente.

Fino al 40\% dei casi c'è abuso di lassativi o di diuretici (possono
arrivare ad assumere una scatola intera di medicinali).

Come avevamo già detto, compiono anche esercizio fisico estenuante al
fine di consumare quante più calorie possibile.

I pazienti diabetici inoltre a proposito di farmaci, hanno uno strumento
potentissimo nonché pericolosissimo: potrebbero evitare di
somministrarsi l'insulina, quindi dimagriscono ma il loro diabete
diventa presto scompensato.

Il peso del paziente può essere normale o fluttuare tra valori di
anoressia e valori di sovrappeso

Per quanto riguarda le \textbf{\emph{manifestazioni internistiche}}
della bulimia nervosa queste interessano:

\begin{itemize}
\item
  \textbf{Apparato gastrointestinale:} l'apparato più compromesso in
  corso di BN è l'apparato gastrointestinale: dolore e discomfort
  addominale, stipsi, colon da catartici per i soggetti che abusano di
  lassativi (il colon ne diventa dipendente), il vomito diventa
  automatico, ci possono essere gastriti e lesioni gastroesofagee a
  causa del vomito continuativo, \emph{sindrome di Mallory-Weiss}
  (rottura esofagea dovuta al vomito), carie ed erosioni dentarie (gli
  odontoiatri sarebbero un'ottima porta di accesso a nuovi casi),
  aumentato volume delle ghiandole salivari che provoca aumeto di volume
  di collo e guance (nel tentativo di alzare il ph della bocca le
  ghiandole secernono molta amilasi e diventano ipertrofiche) che danno
  la ``faccia da luna piena'' che conferisce un aspetto meno emaciato.
\end{itemize}

\begin{itemize}
\item
  Ci sono \textbf{alterazioni metaboliche} perché il vomito causa la
  perdita di sodio e potassio con alterazione elettrolitica che porta a:
  \emph{debolezza muscolare,crampi} e bisogna fare attenzione al cuore
  (allungamento del QT,rischio di \emph{aritmie} da ipopotassemia).
\item
  \textbf{Apparato riproduttivo}:Dopo un po' di oscillazioni del peso,
  anche questi soggetti sviluppano amenorrea. Il corpo non si fida più
  di mettere in moto l'apparato riproduttivo.
\item
  I \textbf{segni di Russell}: faccia dorsale delle dita della mano
  destra, sinistra o di tutte e due che sfrega contro gli incisivi
  quando l'individuo se le mette in gola e sono a contatto con il vomito
  acido, diventano rugose, callose e l'unghia viene via.
\end{itemize}

\textbf{\emph{I CRITERI DIAGNOSTICI DEI BINGE EATING DISORDERS}}

Si caratterizza per \textbf{episodi di ingestione di cibo incontrollata,
però tipicamente non ci sono condotte di eliminazione o di compenso}.

Non è chiaro se c'è un disturbo dell'immagine corporea, ma c'è
ugualmente un certo tipo di stress ed è stato provato che ne soffre il
30\% della popolazione che soffre di obesità, anche se esiste anche
nella popolazione generale.

Qui vengono raggruppati tutti i disturbi dell'alimentazione sotto soglia
che hanno una frequenza degli eventi minore della minima attesa per i
soggetti che soffrono di AN eBN, ci sono i disturbi che affliggono
individui che per controllare il peso vomitano anche senza essersi prima
abbuffati, i disturbi da abbuffate notturne ecc...

\textbf{\emph{TRATTAMENTO DCA:}}

Il trattamento dei DCA è per definizione un \textbf{trattamento
multidisciplinare}, che richiede l'intervento non solo di uno
psichiatra, ma di neuropsicologi, psicologi, internisti, nutrizionisti e
pediatri, e si prefigge come obiettivo ovviamente la risoluzione della
fase acuta di malattia, in cui si hanno sia alterazioni psichiche che
fisiche, le quali vanno sempre trattate assieme, e poi punta alla
risoluzione dei disturbi psichiatrici con riabilitazione e prevenzione
delle ricadute.

L'aspetto fondamentale che va compreso nel trattamento dei DCA è che i
pazienti scelgono questo comportamento piuttosto che subire i sintomi di
una malattia, per cui il medico potrebbe chiedersi se sono
irresponsabili, visto che basterebbe mangiare in maniera adeguata per
stare bene, ma il vero problema è che i soggetti non possono farlo,
perché sono afflitti dal \emph{disturbo dell'immagine corporea}, che in
essi svolge un \emph{vero e proprio effetto ansiolitico ed
antidepressivo}, anzi è \emph{l'unica cosa che dà un senso all'esistenza
del paziente}. Il medico deve quindi sforzarsi di \textbf{empatizzare
col paziente}, creare una minima alleanza terapeutica, indagando con
attenzione la resistenza e l'ambivalenza al trattamento, perché in ogni
paziente c'è una parte che desidererebbe curarsi ed una che invece si
oppone, vuole mantenere il disturbo dell'immagine corporea e non vuole
dipendere dal medico. Fondamentale diventa quindi il riuscire a
mantenere ed aumentare la motivazione al trattamento, cioè spiegare al
paziente che quando sarà guarito non ci sarà più nessun disturbo
dell'immagine corporea e potrà stare comunque bene senza di esso. In
genere si dovrebbe cercare di stipulare una sorta di
``\textbf{\emph{contratto terapeutico}}'': programma
cognitivo-comportamentale prende il nome di riabilitazione
psico-nutrizionale; \emph{viene fissato un certo peso, deciso dal
nutrizionista assieme allo psichiatra} (in genere l'85\% del peso
corporeo normale), \emph{che il paziente deve raggiungere entro un
mese}, e \emph{si lavora gradualmente anche sui cibi, cercando di
riportare, passo dopo passo, il paziente ad un'alimentazione corretta}
tramite una sorta di ``\textbf{rieducazione alimentare}''. Per quanto
riguarda poi il \textbf{\emph{ruolo dei farmaci}}, questi differiscono
leggermente a seconda che ci si trovi di fronte ad un caso di AN o di
BN: per l'\emph{AN} non ci sono farmaci capaci di ``sconfiggere'' il
disturbo, a dimostrazione che non si tratta di un disturbo psicotico,
altrimenti gli anti-psicotici avrebbero un qualche effetto, quindi
\emph{in fase acuta di AN non si usano psicofarmaci}, ma si fa solo una
terapia nutrizionale volta a ristabilire le condizioni corporee del
paziente e far regredire la malnutrizione; quando quest'ultima è stata
trattata si può accedere con maggior facilità alla vera personalità del
paziente (ricordare sempre che la malnutrizione tende ad esasperare gli
aspetti psichiatrici del soggetto, rendendolo per certi versi
``inaccessibile'' allo psichiatra) e nella terapia di mantenimento,
volta alla stabilizzazione del peso raggiunto e alla prevenzione delle
ricadute, diventa fondamentale la psicoterapia associata all'uso di
\textbf{farmaci antidepressivi}, ma bisogna stare attenti a \emph{non
cadere nell'errore di dare un farmaco che aumenti l'appetito}, prima di
tutto perché questi pazienti hanno un appetito ed una fame normale,
anche se li negano, e poi si sentirebbero traditi dal medico, per cui si
sceglie in genere di usare gli \textbf{\emph{SSRI}}, tra cui il più
adatto è la \textbf{fluoxetina}, che tra tutti dà il minor incremento
dell'appetito, oppure anche la \textbf{sertralina}.

Per il trattamento della \emph{bulimia nervosa}, invece, bisogna prima
capire bene di che tipo di BN si tratta, poi si trattano le condotte
anomale ed i disturbi associati. Il trattamento della bulimia si
sviluppa quindi in 3 fasi principali:

\begin{enumerate}
\def\labelenumi{\arabic{enumi}.}
\item
  \textbf{\emph{Fase di Valutazione Personale Pre-Trattamento}}: è volta
  a ricercare eventuali disturbi psichiatrici associati, e se sono
  presenti sintomi di BN multi-impulsiva è opportuno ricorrere a degli
  psicofarmaci, più nello specifico agli \textbf{SSRI} e agli
  \textbf{stabilizzatori dell'umore}, mentre sono da evitare i TCA
  perché aumentano l'appetito ed il litio, che ha un basso indice
  terapeutico.
\item
  \textbf{\emph{Fase di Trattamento delle Condotte Alimentari
  Patologiche}}: si fa prima un counseling nutrizionale, seguita da
  terapia cognitivo-comportamentale (CTB) e dall'uso di psicofarmaci, in
  particolare la \textbf{fluoxetina} a dosi di 60-80 mg, capace di
  ridurre nel breve periodo il discontrollo impulsivo nei confronti del
  cibo.
\item
  \emph{\textbf{Fase} \textbf{di Trattamento a Lungo Termine dei
  Disturbi Psichiatrici Associati}}: prevede il ricordo alla
  psicoterapia individuale ed eventualmente anche ad una psicofarmaco
  terapia di mantenimento.
\end{enumerate}

Attenzione va posta all'uso dell'\emph{olanzapina}, che sarebbe da
evitare, perché se da un lato riduce la preoccupazione per l'immagine
corporea e l'iperattività ritualistica, dall'altro determina un notevole
aumento dell'appetito ed allunga il QT, peraltro spesso già allungato
dall'ipokaliemia.

Indipendentemente dal tipo di DCA che ci si trovi ad affrontare, è poi
molto importante stabile l'\emph{appropriato livello di cura} per il
paziente: in generale i pazienti con DCA dovrebbero essere
\textbf{trattati a livello ambulatoriale}, cioè il livello di cura meno
restrittivo, così da non sottoporre il paziente ad eccessivi stress e
rafforzare l'alleanza terapeutica, tuttavia ciò non è sempre possibile,
anzi spesso vi sono delle condizioni che richiedono il ricovero, che può
essere di due tipi: il \emph{ricovero urgente} e il \emph{ricovero
riabilitativo}.

Il \textbf{ricovero urgente} viene effettuato o per \emph{motivi
psichiatrici} o per \emph{motivi internistici}: nel primo caso il
soggetto sta avendo una crisi psichiatrica, ed è una situazione tipica
del pazienti con BN, che hanno frequentemente anche disturbo borderline
di personalità ed altre comorbilità, e in questi casi il ricovero è in
ambiente psichiatrico, mentre nel secondo caso, più comune per i
soggetti con AN, il ricovero è effettuato quando è necessaria una
nutrizione artificiale o un attento monitoraggio per prevenire
un'ulteriore perdita di peso, e viene effettuato in strutture di
medicina interna.

Il \textbf{ricovero riabilitativo} viene invece effettuato al di fuori
di un'urgenza medica o psichiatrica, e viene generalmente richiesto
\emph{se si ha ragione di ritenere che a livello ambulatoriale il
paziente non riesca a tenersi al trattamento}, per cui lo si ricovera in
una struttura (ad esempio Villa Maria Luigia a Parma), in cui si cucina
e si mangia tutti assieme ed il controllo è maggiore.

\textbf{\emph{DECORSO E FOLLOW-UP DEI DCA}}

Il decorso dei DCA è alquanto complesso, poiché queste forme tendono
spesso ad evolvere l'una nell'altra, sino allo sviluppo di una
condizione detta disturbo alimentare atipico o DCA NAS, che nella
maggior parte dei casi va poi incontro a risoluzione. In generale, nel
follow-up a 5 anni dei DCA si riscontra la classica ``\textbf{regola
dell'1/3}'':

\begin{itemize}
\item
  1/3 dei pazienti dimostra un outcome buono,
\item
  1/3 un outcome intermedio
\item
  1/3 un outcome invariato, all'interno del quale si riscontra peraltro
  quel 5-20\% di mortalità legata ai DCA.
\end{itemize}

Nell'\emph{anoressia nervosa}, il decorso è spesso complicato da un
30-50\% di drop-out in fase acuta e da un altro 30-50\% di ricadute
entro un anno, tuttavia bisogna tenere presente che i soggetti ad
esordio precoce (entro i 13 anni) hanno un decorso autolimitantesi e
sono anche quelli che guariscono meglio, poiché in questi soggetti il
DCA è legato più ad un'azione culturale-ambientale che ad una vera e
propria predisposizione genetica, sebbene vada precisato che anche in
questo gruppo un 20\% dei casi ha un decorso intrattabile e cronico. Più
comunemente, indipendentemente dall'età d'esordio, il decorso ha un
andamento fluttuante, cioè il disturbo dell'immagine corporea non
scompare del tutto ma permane sotto forma di preoccupazioni.

Nella \emph{bulimia nervosa}, invece, il follow-up a 5 anni mostra
sintomi significativi nel 30\% circa dei casi, e sintomi prognostici
negativi sono una pregressa storia di obesità nell'infanzia e la scarsa
autostima, sebbene vada tenuto a mente che all'interno delle sindromi
bulimiche le forme multi-impulsive sono le più difficili da trattare ed
hanno elevate percentuali di drop-out e ricadute a lungo termine.

\end{document}

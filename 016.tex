\section{Trattamento Sanitario Obbligatorio (TSO)}

L'articolo 32 della Costituzione Italiana afferma che ``\emph{La
Repubblica tutela la salute come diritto fondamentale dell'individuo ed
interesse della collettività, garantendo le cure anche agli
indigenti}''. Tuttavia, sempre nella Costituzione si afferma che
``\emph{Nessuno può essere obbligato ad un determinato trattamento
sanitario se non per disposizione di legge. La legge non può in nessun
caso violare i limiti imposti dal rispetto della persona umana}.''

Il bene della vita, della salute e dell'integrità fisica e psichica sono
dunque oggetto dei diritti della persona, e si tratta di diritti
naturali essenziali ed assoluti, e come tali indisponibili,
irrinunciabili, non trasmissibili e non espropriabili. Da ciò deriva
quindi che gli accertamenti ed i trattamenti sanitari sono di norma
volontari, ovvero basati sul consenso, ed i provvedimenti obbligatori,
disposti secondo la legge, devono comunque rispettare la dignità della
persona, prestando sempre garanzie e diritti ai cittadini. Pertanto, per
poter procedere con un trattamento medico o anche con accertamenti
diagnostici rilevanti è necessario ottenere il \textbf{\emph{consenso}}
del paziente, il quale dev'essere:

\begin{itemize}
\item
  \textbf{Personale};
\item
  \textbf{Libero e Spontaneo};
\item
  \textbf{Consapevole ed Informato};
\item
  \textbf{Attuale};
\item
  \textbf{Gratuito};
\item
  \textbf{Manifesto}.
\end{itemize}

Il consenso non coincide con l'\textbf{assenso}, cioè col semplice
benestare, l'acconsentire passivo a qualcosa; il consenso richiede
infatti un incontro di volontà, di alleanze e di partecipazione attiva.
Il consenso informato è una decisione condivisa all'interno di una
relazione di cura fondata sulla fiducia e la chiarezza comunicativa.

In ambito psichiatrico, tuttavia, i disturbi mentali possono determinare
assenza di consapevolezza di malattia e, sulla base di rappresentazioni
e valutazioni gravemente alterate, vengono rifiutati interventi sanitari
necessari ed urgenti, per cui in questi casi si può effettuare un
trattamento sanitario obbligatorio (TSO), il quale costituisce l'estremo
mezzo per rendere effettivo il diritto alla salute di un individuo con
una grave patologia psichica di cui non è consapevole.

Il TSO, in ogni caso, è effettuato non come misura di ``difesa
sociale'', ma come recupero dell'individuo alla collettività, inoltre il
TSO deve sempre essere accompagnato dalla ricerca del consenso di chi vi
è obbligato.

Per poter effettuare un TSO, devono essere quindi soddisfatte \emph{3
condizioni}:

\begin{itemize}
\item[1.]
  \textbf{Dev'essere presente un'alterazione psichica tale} (cioè
  talmente grave) \textbf{da richiedere interventi terapeutici urgenti,
  ovvero non differibili nel tempo};
\item[2.]
  \textbf{Gli interventi terapeutici non vengono accettati dall'infermo}
  (ed il rifiuto è strettamente legato al disturbo mentale);
\item[3.]
  \textbf{Non sussistono le condizioni e le circostanze per attuare
  idonei e tempestivi interventi in ambito extra-ospedaliero}.
\end{itemize}

\subsection{Procedura del TSO}

\begin{itemize}
\item
  Un medico (qualsiasi medico) visita la persona e, se ritiene che
  sussistano le tre condizioni di legge, prepara una proposta motivata
  di TSO in triplice copia con firma autografa.
\item
  Un secondo medico dell'Unità sanitaria locale visita la persona e
  redige il documento di convalida della proposta, sempre in triplice
  copia con firma autografa.
  
N.B.: In questa fase è necessario anche documentare tutti i tentativi
fatti per assicurare il consenso e la partecipazione da parte di chi vi
è obbligato, anche per far fronte a eventuali contestazioni e ricorsi
all'autorità amministrativa e al tribunale territoriale.

\item
  I documenti di proposta e convalida del TSO vengono recapitati,
  tramite la Polizia Municipale, al Sindaco, il quale emette l'ordinanza
  di TSO, che viene eseguita sempre dalla polizia municipale di concerto
  col personale sanitario fino al ricovero, che può avvenire solo presso
  l'SPDC.
\end{itemize}

Il TSO ha una durata massima di \emph{7 giorni}; se vengono meno le
condizioni, il TSO, può comunque essere revocato prima dal direttore del
SPDC, che lo comunica al sindaco, oppure, se persistono le condizioni,
il direttore del SPDC può chiedere al sindaco che ha emesso l'ordinanza
di TSO la proroga (eventualmente ripetuta più volte) motivate,
indicandone la durata presunta.

\subsection{Garanzie del TSO}

L'ordinanza del TSO va notificata alla persona che vi è sottoposta, la
quale conserva tutti i suoi diritti civili e politici, compreso quello
di comunicare, e deve essere sempre accompagnato dalla ricerca del
consenso e dal rispetto della dignità della persona. Il TSO, inoltre,
non viene annotato in nessun documento, né nel casellario o non comporta
la perdita della patente eccetera. Inoltre, entro 48 ore, il sindaco
deve inviare l'ordinanza al Giudice Tutelare, il quale, entro altre 48
ore, assunte le informazioni e gli eventuali accertamenti, provvede con
decreto motivato a convalidare o non convalidare il provvedimento del
sindaco. Il giudice tutelare può adottare provvedimenti urgenti per
amministrare il patrimonio della persona, e chi è sottoposto a TSO o
chiunque vi abbia interesse, può fare ricorso presso il tribunale
competente, che decide entro 30 giorni dal ricorso.

\subsection{Condizioni che non richiedono il TSO}

Esistono delle condizioni che, presentandosi urgenti e complesse, non
richiedono l'attivazione delle procedure del TSO, in quanto la persona è
incapace di esprimersi rispetto alla proposta di cura, e in questo caso
la priorità è la tutela della salute della persona, interessata da gravi
compromissioni dello stato di coscienza e dall'impossibilità di
esprimere o negare il consenso (ad esempio per traumi, intossicazioni,
delirium e demenze). In questi casi, infatti, si applica l'articolo 54
del codice penale, cioè lo stato di necessità.

\subsection{Limiti del TSO}

Il TSO \emph{non può essere effettuato per obbligare una persona ad
effettuare interventi diagnostici e terapeutici} per altre patologie
internistiche o chirurgiche, anche se indicate o prescritte dai medici o
con un alto rapporto benefici/rischi; in questi casi, molto complessi,
può essere necessario o opportuno coinvolgere il magistrato. Altri
limiti del TSO risiedono nel fatto che \emph{non esistono leggi che
obbligano la persona con disturbi mentali a seguire le cure a
domicilio}, e in questi casi è necessaria la ricerca del consenso, e
\emph{non rientra nelle competenze dei medici e degli psichiatri la
protezione dei familiari rispetto ai conviventi violenti}, e in questi
casi solo un giudice, in base agli articoli 342 bis e ter del codice
civile può disporre l'allontanamento della persona violenta.

\subsection{TSO Extra-Ospedaliero e TSO nei Minori}

Se sussistono le prime due condizioni per il TSO ma non la terza può
essere proposto un TSO extra-ospedaliero, che di solito viene usato solo
per l'effettuazione di terapie depot di pazienti in cura ai CSM.

Nei minori, invece, va sempre ricercato il consenso sia del minore che
della famiglia, e in questi casi si possono riscontrare due condizioni:

\begin{itemize}
\item
  \emph{Minore bisognoso di cure urgenti e consenziente al trattamento,
  ma genitori contrari, e in questi casi è necessario ricorrere al TSO};
\item
  \emph{Minore bisognoso di cure urgenti, non consenziente, e genitori
  favorevoli, in questo caso non servirebbe il TSO formale, ma il
  rifiuto del minore dev'essere valutato e documentato}.
\end{itemize}

In entrambi i casi è utile informare il Tribunale per i Minori, in
particolare se l'ambiente familiare non è adeguato (\textbf{articolo 403
del codice civile}).

\subsection{Accertamento Sanitario Obbligatorio (ASO)}

L\textbf{'ASO} è un \emph{accertamento sanitario obbligatorio}
effettuato quando si ha il fondato sospetto sulla prima condizione di
legge prevista per poter intervenire in forma obbligatoria, e in cui
l'approfondimento diagnostico necessario non viene accettato dal
paziente. Si tratta comunque di un intervento eccezionale, che va sempre
preceduto ed accompagnato dalla ricerca del consenso. La proposta di ASO
può essere effettuata dal medico che ha constatato la presenza delle
condizioni sopracitate e previste dalla legge, e in questo caso, ai fini
dell'emissione dell'ordinanza di ASO da parte del sindaco basta la sola
certificazione medica di proposta, in triplice copia, contenente le
motivazioni dettagliate che sostengono la richiesta; a seguito
dell'ordinanza del sindaco la polizia municipale ed il personale
sanitario accompagnano la persona nel luogo stabilito per la visita, e
deve assicurare la sua presenza fino al termine della stessa.

La legge, inoltre, prevede che l'ASO non possa avvenire in regime di
ricovero, ma può essere effettuato presso il PS, il CSM o in
ambulatorio, e l'esito può essere notevolmente diverso, ad esempio si
può avere una restituzione al medico di medicina generale, la presa in
cura al CSM, il ricovero volontario o anche un TSO, ma in ogni caso
l'esito dell'ASO va comunicato al Sindaco.

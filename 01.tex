\documentclass[]{article}
\usepackage{lmodern}
\usepackage{amssymb,amsmath}
\usepackage{ifxetex,ifluatex}
\usepackage{fixltx2e} % provides \textsubscript
\ifnum 0\ifxetex 1\fi\ifluatex 1\fi=0 % if pdftex
  \usepackage[T1]{fontenc}
  \usepackage[utf8]{inputenc}
\else % if luatex or xelatex
  \ifxetex
    \usepackage{mathspec}
  \else
    \usepackage{fontspec}
  \fi
  \defaultfontfeatures{Ligatures=TeX,Scale=MatchLowercase}
\fi
% use upquote if available, for straight quotes in verbatim environments
\IfFileExists{upquote.sty}{\usepackage{upquote}}{}
% use microtype if available
\IfFileExists{microtype.sty}{%
\usepackage{microtype}
\UseMicrotypeSet[protrusion]{basicmath} % disable protrusion for tt fonts
}{}
\usepackage[unicode=true]{hyperref}
\hypersetup{
            pdfborder={0 0 0},
            breaklinks=true}
\urlstyle{same}  % don't use monospace font for urls
\IfFileExists{parskip.sty}{%
\usepackage{parskip}
}{% else
\setlength{\parindent}{0pt}
\setlength{\parskip}{6pt plus 2pt minus 1pt}
}
\setlength{\emergencystretch}{3em}  % prevent overfull lines
\providecommand{\tightlist}{%
  \setlength{\itemsep}{0pt}\setlength{\parskip}{0pt}}
\setcounter{secnumdepth}{0}
% Redefines (sub)paragraphs to behave more like sections
\ifx\paragraph\undefined\else
\let\oldparagraph\paragraph
\renewcommand{\paragraph}[1]{\oldparagraph{#1}\mbox{}}
\fi
\ifx\subparagraph\undefined\else
\let\oldsubparagraph\subparagraph
\renewcommand{\subparagraph}[1]{\oldsubparagraph{#1}\mbox{}}
\fi

% set default figure placement to htbp
\makeatletter
\def\fps@figure{htbp}
\makeatother


\date{}

\begin{document}

Psicopatologia

Col termine \textbf{\emph{psicopatologia}} si intende lo \emph{studio
dei singoli fenomeni morbosi psichici}, \emph{indipendentemente dai vari
disturbi clinici in cui essi ricorrono}, ed è una disciplina la cui
origine che viene tradizionalmente fatta risalire al 1913, data di
pubblicazione del testo ``Psicopatologia Generale'' di Jaspers.

La psicopatologia si occupa quindi dei fenomeni morbosi non solo nella
loro rilevazione e descrizione, ma anche nello studio dei modelli
psiconeurobiologici che possono esserne alla base, e consente pertanto,
da un lato, di edificare dei quadri sindromici, e dall'altro di cogliere
ed analizzare questi stessi fenomeni indipendentemente dal quadro
morboso a cui appartengono.

Oggetti di studio della psicopatologia sono quindi i seguenti elementi
della psiche:

\begin{itemize}
\item
  \textbf{Percezione};
\item
  \textbf{Attenzione};
\item
  \textbf{Memoria};
\item
  \textbf{Pensiero};
\item
  \textbf{Intelligenza};
\item
  \textbf{Coscienza};
\item
  \textbf{Affettività};
\item
  \textbf{Istintualità};
\item
  \textbf{Volontà} (\textbf{Psicomotilità}).
\end{itemize}

\textbf{Percezione}

La \textbf{\emph{percezione}} è un'\emph{attività psichica conoscitiva},
a \emph{carattere in parte recettivo e in parte costruttivo}, che
consente di cogliere la realtà (sensazioni attuali elaborate dagli
organi di senso) e di strutturarla sulla base dei dati dell'esperienza.

La psicopatologia della percezione comprende:

\begin{itemize}
\item
  \textbf{Le alterazioni dell'intensità e della qualità delle
  percezioni};
\item
  \textbf{Le anomalie dei caratteri percettivi};
\item
  \textbf{La dissociazione delle percezioni};
\item
  \textbf{Il falsamento delle percezioni}.
\end{itemize}

Le \textbf{alterazioni dell'intensità e della qualità delle percezioni}
sono essenzialmente delle deformazioni quantitative dei caratteri
sensoriali di una percezione, che includono fenomeni di accentuazione o
diminuizione, identificate che \emph{iperestesia} o \emph{ipoestesia}, o
anche come cambiamenti di una qualità percettiva elementare, come il
colore (\emph{coloropsia}), il volume (\emph{macro/micropsia}) e della
collocazione spaziale (\emph{porropsia}).

Le \textbf{anomalie dei caratteri percettivi} includono invece delle
\emph{modificazioni delle qualità accessorie che accompagnano i
percetti}, come \emph{sentimenti di familiarità o estraneità},
\emph{sentimenti di realtà o irrealtà} ed anche \emph{alterazioni dei
sentimenti di piacevolezza o spiacevolezza}.

La \textbf{dissociazione delle percezioni}, invece, consiste
nell'\emph{incapacità di fondere in un unico evento percezioni
simultanee provenienti dallo stesso oggetto}.

Molto importanti sono anche i \textbf{falsamenti delle percezioni}, che
includono le \emph{illusioni}, le \emph{allucinazioni}, le
\emph{pseudoallucinazioni} e le \emph{allucinosi}; le illusioni sono
percezioni distorte della realtà con caratteri non rispondenti alle sue
reali proprietà fisiche, e sono il prodotto inadeguato dell'interazione
tra i dati sensoriali e psicologici soggettivi. In generale, le
\textbf{illusioni} hanno uno scarso rilievo clinico, ritrovandosi come
sintomi accessori in diversi quadri psicopatologici. Le
\textbf{allucinazioni} sono invece delle percezioni senza oggetto, che
rappresentano qualcosa di completamente nuovo e spontaneo, svincolato da
qualsiasi apporto sensoriale, e sono generalmente dei fenomeni
dispercettivi di comune riscontro nelle psicosi, in base all'organo
sensoriale a cui sono riferite possono essere:

\begin{itemize}
\item
  \textbf{Allucinazioni Visive}; tipiche delle sindromi psichiatriche
  organiche e tossiche, come il delirium tremens, le psicosi organiche o
  le intossicazioni da droghe.
\item
  \textbf{Allucinazioni Uditive}; che possono consistere in
  \emph{allucinazioni parafasiche} (il paziente sente parole indistinte
  o appena bisbigliate), \emph{allucinazioni neologistiche} (in cui il
  paziente sente delle parole nuove o incomprensibili),
  \emph{allucinazioni imperative}, \emph{allucinazioni teleologiche},
  \emph{allucinazioni con voci dialoganti} o \emph{allucinazioni con
  voci dialoganti}. In ogni caso, questo tipo di allucinazioni sono
  tipiche della parafrenia, delle psicosi deliranti, di quelle
  confusionali e della schizofrenia.
\item
  \textbf{Allucinazioni Olfattive} e \textbf{Gustative}; sono
  solitamente a contenuto spiacevole, si ritrovano negli stati
  confusionali, ma più raramente si riscontrano nella mania con
  esperienze erotiche e mistiche, nella melanconia e nelle psicosi
  croniche, sebbene possano contrassegnare anche le crisi comiziali
  connesse a focalità temporale (le cosiddette \emph{crisi del giro
  uncinato}).
\item
  \textbf{Allucinazioni Somatiche}; sono turbe della percezione
  somato-estesica, sono quindi difficilmente classificabili in quanto la
  percezione del corpo è un fenomeno estremamente complesso, sia per
  l'assenza di un organo ben definito della sensibilità corporea, sia
  perché il corpo è contemporaneamente soggetto ed oggetto del fenomeno
  percettivo. All'interno di questo gruppo troviamo le allucinazioni
  cenestesiche, le allucinazioni tattili-termiche e le allucinazioni
  chinestesiche-motrici: le \emph{allucinazioni cenestesiche} sono delle
  percezioni abnormi di tutto il corpo o di parti di esso, e si possono
  manifestare come supposte aggressioni corporali allucinatorie, e
  possono anche spingere il soggetto ad adottare mezzi di protezione di
  varia natura, essendo comuni nella schizofrenia, nelle gravi forme
  depressive e nelle psicosi organiche. Le \emph{allucinazioni tattili}
  consistono invece in percezioni di formicolio sulla pelle, mentre
  quelle \emph{termiche} si manifestano con folate di vento caldo o
  freddo, nonché bruciature, sebbene esistano anche forme più
  specifiche, le \emph{allucinazioni idriche} (sensazione di bagnato) o
  \emph{aptiche} (sensazioni di scosse elettriche). Infine, le
  \emph{allucinazioni chinestesiche-motrici} si caratterizzano per il
  fatto che il soggetto avverte un senso di movimento nonostante sia
  completamente immobile, oppure percepisce delle contrazioni muscolari
  senza spostamento reale, oppure può sentirsi costretto a compiere
  azioni o sente un'inibizione che gli impedisce di muoversi.
\end{itemize}

Le \textbf{pseudoallucinazioni}, invece, sono fenomeni che, pur essendo
dotati di tutti i caratteri delle allucinazioni, non vengono obietti
vati nel mondo esterno ma appaiono nello spazio interno soggettivo: il
paziente li vive come fenomeni psichici che, per quanto strani ed
inspiegabili, non hanno alcun rapporto con la realtà esterna; queste
pseudoallucinazioni sono essenzialmente uditive, ma possono anche essere
visive, e sono spesso vissuti come estranei ed imposti, essendo tipici
della schizofrenia.

Le \textbf{allucinosi}, infine, sono fenomeni allucinatori visivi o
uditivi dotati di notevole estesia sensoriale, talora a carattere
scenico e colorito, proiettato nello spazio oggettivo esterno ma che non
viene vissuto dal soggetto in termini di uno stimolo reale essendo
riconosciuto come fenomeno patologico. Si osserva più frequentemente
nella patologia organica cerebrale, dovuta a lesioni temporali ed
occipitali, o su base tossica. Particolare rilievo va prestata alla
cosiddetta ``\textbf{\emph{allucinosi alcolica}}'', un termine in realtà
improprio, che si riferisce ad una sindrome allucinatoria che insorge in
alcolisti cronici, caratterizzata da allucinazioni, prevalentemente o
esclusivamente uditive, a contenuto spiacevole (voci che parlano tra
loro di lui, lo insultano o lo minacciano), vissute in una condizione di
lucidità e di coscienza, a cui si correlano deliri a contenuto
persecutorio o di colpa.

\textbf{Attenzione}

L'\textbf{\emph{attenzione}} è una \emph{funzione psichica atta a
focalizzare e dirigere la coscienza su specifici contenuti}, sia
oggettivi che soggettivi, fra quanti compongono il suo campo fenomenico.

I disturbi dell'attenzione includono l'\textbf{ipoprosessia} (detta
anche semplicemente ``distrazione''), cioè una diminuzione delle
capacità attentive che si osserva in tutte quelle condizioni che sono
associate ad una diminuzione del livello di vigilanza, come la
sonnolenza, l'ebbrezza o le psicosi tossiche, ma sono alterate anche
negli stati melanconici e nelle insufficienze mentali. Altre alterazioni
dell'attenzione sono anche l'\textbf{aprosessia}, cioè la totale
mancanza di attenzione, che è tipica degli stati confusionali, e
l'\textbf{iperprosessia}, in cui invece si osserva un aumento
dell'attenzione, che si rileva nelle intossicazioni da allucinogeni,
nelle nevrosi fobiche e nelle psicosi deliranti.

\textbf{Memoria}

Quello della ``\textbf{\emph{memoria}}'' è un termine a cui sono
assegnati diversi significati, ad esempio può indicare i
\emph{meccanismi che sottendono i processi mnemonici}, ma può stare ad
indicare anche \emph{i contenuti dell'esperienza depositati nel sistema
mnesico}, o anche la \emph{prestazione mnesica}, cioè la capacità di
ricordare.

Indipendentemente dal significato, la memoria è la
\textbf{\emph{capacità di riprodurre un'esperienza passata}}, di
\textbf{\emph{riconoscerla e localizzarla nel tempo e nello spazio}}.
Nella memoria si distinguono due momenti fondamentali, cioè la
\emph{fissazione}, che consente di registrare i contenuti di coscienza,
di conservare di essi tracce stabili (engrammi mnesici), mentre la
seconda fase è la \emph{rievocazione}, che permette di ricondurre alla
coscienza determinati contenuti. Nella memoria, inoltre, si possono
distinguere due tipi principali, cioè la \textbf{memoria a breve
termine}, durante la quale si formano delle tracce mnesiche, ma non il
loro consolidamento, e la \textbf{memoria a lungo termine}, in cui le
tracce si consolidano così da poter perdurare nel tempo.

I disturbi quantitativi della memoria sono essenzialmente
l'\emph{ipermnesia}, l'\emph{amnesia} (retrograda o anterograda) e la
\emph{confabulazione} (creazione di falsi ricordi a riempimento di
lacune mnesiche del passato). Questa neoproduzione, accurata e
fantasiosa, attinge a diversi frammenti mnemonici dell'esperienza
passata e agli stimoli dell'ambiente. Le confabulazioni sono tipiche
della sindrome di Korsakoff, della demenza e delle sindromi
post-traumatiche. Tutti questi disturbi sono di tipo quantitativo,
mentre i disturbi qualitativi della memoria sono le \emph{allomnesie}
(illusioni della memoria) e le \emph{pseudomnesie} (allucinazioni della
memoria), che includono i falsi riconoscimenti ed i falsi ricordi.

\textbf{Pensiero}

Il \textbf{\emph{pensiero}} è quell'attività operativa della psiche che,
attraverso processi di associazione, correlazione, integrazione,
astrazione e simbolizzazione dei dati informativi (percezioni e
rappresentazioni), permette la \emph{valutazione della realtà e la
formulazione di idee, astrazioni intellettuali, non riconducibili a
stimoli attuali né ad alcun oggetto concreto}. L'\textbf{idea} è un
modello generale di un oggetto che non si riferisce ad un oggetto in
particolare.

Il giudizio di realtà si fonda, oltre che sul confronto e l'integrazione
dei dati forniti dall'esperienza, sui rapporti di causalità empirica
esistenti tra singoli dati e su una rete di valori e significati
codificati dal sistema sociale e culturale. Il pensiero non si sviluppa
soltanto secondo la logica, ma anche in armonia a fattori soggettivi,
sia di ordine pratico che affettivo.

I disturbi del pensiero sono suddivisibili in:

\begin{itemize}
\item
  \textbf{\emph{Disturbi Formali del Pensiero}};
\item
  \textbf{\emph{Disturbi del Contenuto del Pensiero}}.
\end{itemize}

Tra i disturbi formali del pensiero sono inclusi:

\begin{itemize}
\item
  L'\textbf{Accelerazione Ideica}, cioè un'accelerazione del pensiero
  caratterizzata da \emph{eloquio scorrevole}, \emph{abbondante} e
  \emph{irrefrenabile} (\textbf{logorrea}), giochi di parole e battute
  di spirito. Nel suo grado estremo si ha la cosiddetta \emph{fuga delle
  idee}, coi contenuti ideativi che si succedono con straordinaria
  celerità, associati tra loro in modo superficiale. L'accelerazione
  ideica di grado modesto si trova in condizioni di ebbrezza e sotto
  l'effetto di psicostimolanti. In ogni caso, la fuga delle idee e la
  logorrea sono tipiche dell'eccitamento maniacale.
\item
  Il \textbf{Rallentamento del Pensiero}, in cui i contenuti ideativi
  sono scarsi e poveri, l'associazione è lenta, l'elaborazione stentata
  e l'espressione difficoltosa. Il paziente riferisce spesso la
  \emph{sensazione di ristagno del pensiero}, che è fissato su pochi
  temi ed è incapace di concretizzarsi in un prodotto finale. Il
  rallentamento del pensiero si associa ad un'\emph{inerzia
  dell'attività motoria e dell'espressività mimica}. Nei casi più gravi
  si può arrivare ad un'inibizione completa che prende il nome di
  \textbf{arresto} \textbf{psicomotorio}, ed è tipico delle forme più
  avanzate di disturbo depressivo.
\item
  La \textbf{Prolissità}, cioè il pensiero che raggiunge la sua meta
  soltanto in modo indiretto, attraverso l'interferenza di idee
  secondarie che ne pregiudicano il decorso.
\item
  La \textbf{Circostanzialità}, che si esprime con eloquio pedante e
  barocco, ricco di dettagli irrilevanti sviluppati in modo abnorme,
  quasi venisse meno la capacità di distinguere l'essenziale dal
  marginale.
\item
  La \textbf{Perseverazione}, cioè una persistenza nella coscienza di
  pensieri ripetuti in modo continuativo ed acritico, che non vengono
  rimossi da nuovi stimoli ed esperienze.
\item
  Il \textbf{Pensiero Dissociato}, cioè un \emph{pensiero frammentario},
  \emph{privo di logicità}, \emph{bizzarro} e \emph{sconclusionato}, in
  cui vengono meno i comuni caratteri associativi fra le singole idee,
  ed è tipico delle psicosi schizofreniche. Si caratterizza per fenomeni
  di fusione delle idee, iperinclusioni, digressioni o deragliamenti del
  pensiero ed intoppi del pensiero.
\item
  Il \textbf{Pensiero Incoerente}, in cui l'alterazione dell'ideazione è
  ricondotta ad una modificazione dello stato di coscienza con
  abbassamento dello stato di vigilanza e labilità dell'attenzione,
  tipico delle psicosi confusionali organiche.
\item
  Il \textbf{Pensiero Anancastico}, che è un pensiero, una
  rappresentazione (\emph{ossessione}) o una spinta impellente
  all'azione (\emph{compulsione}) non intenzionale e irrazionale,
  suscettibile alla critica da parte del paziente, che non riesce a
  liberarsene e lo vive con sensazione di fastidio ed angoscia. I
  contenuti di coscienza sono riconosciuti dall'individuo come propri,
  ma considerati inaccettabili e rifiutati, perché vissuti come estranei
  all'Io, che ne viene ostacolato nella sua libera espressione, e contro
  a questi disturbi si mette in moto uno psichismo di difesa, il
  paziente cerca di respingerli e di resistere loro. Le ossessioni e le
  compulsioni sul piano comportamentale si traducono in atti compulsivi
  con significato di controllo e di difesa magica nei confronti del
  pensiero parassita.
\end{itemize}

I disturbi del contenuto del pensiero, invece, comprendono le
\textbf{idee prevalenti} e i \textbf{deliri}: le \textbf{\emph{idee
prevalenti}}, sono idee o complessi di idee affini nel contenuto,
sorrette da un fondo affettivo molto intenso, che predominano su ogni
altro pensiero, costituendo una caratteristica temporanea o permanente
della personalità del soggetto di cui ne ispirano la condotta. Si
elaborano in genere su eventi possibili o reali sono comprensibili nella
loro motivazione affettiva e sono ancora accessibili alla critica,
avendo un contenuto connesso alla realtà oggettiva con corretti rapporti
di causalità logica.

Il \textbf{\emph{delirio}}, invece, è per Jaspers, `` un giudizio
patologicamente falsato'', che si caratterizza per l'\emph{assoluta
certezza soggettiva}, una \emph{convinzione incorreggibile e l'assurdità
del contenuto}. Il delirio, quindi, è un'idea o un sistema di idee,
caratterizzate da modalità genetiche e formali assolutamente peculiari,
estranee quindi nelle premesse e nelle conclusioni al mondo comune e
sostenute da un convincimento acritico; come evidenziato da Schneider,
\emph{ciò che caratterizza il delirio, tuttavia, non è tanto il
contenuto, quanto il modo in cui il contenuto è percepito e la forma con
cui si esprime}. Nel delirio, il soggetto vive in una modalità
esperienziale caratterizzata dal venir meno di quella trama di rapporti
col mondo, che rende quest'ultimo comune e comunicabile, condivisibile e
quindi vivibile. I significati delle cose sono comuni solo quando
assumono un valore, che trascendendo l'esperienza del singolo, li rende
quindi comunicabili.

Il delirio può essere distinto in \textbf{delirio lucido}, quando è
vissuto con uno stato di coscienza vigile, e \textbf{delirio}
\textbf{confuso} quando invece lo stato di coscienza è alterato ( deliri
febbrili, tossici e dismetabolici); il delirio può anche essere
\textbf{sistematizzato}, quando ha una buona organizzazione interna, con
articolazione coerente tra i diversi contenuti, oppure \textbf{non
sistematizzato}, quando invece è frammentario, costituito da idee
isolate, semplici ed incostanti, non correlate tra loro.

In virtù dei loro caratteri genetici e formali, i deliri sono suddivisi
in:

\begin{itemize}
\item
  \textbf{\emph{Deliri Primari}};
\item
  \textbf{\emph{Deliri Secondari}}.
\end{itemize}

I \textbf{deliri primari} sono indipendenti da qualsiasi esperienza
psichica, si presentano come \emph{fenomeni primari inderivabili},
\emph{psicologicamente indeducibili e quindi incomprensibili}. Si deve
però tenere a mente che incomprensibile non è il contenuto del delirio,
ma il modo in cui tale convinzione si instaura che appare non intuibile,
che sfugge alla capacità di immedesimazione e comprensione. Si tratta di
eventi che esulano da qualsiasi linea di causalità psicologica, ed i
deliri primari sono spesso preceduti o accompagnati nel loro formarsi da
uno \emph{stato d'animo pre-delirante}, detto anche \textbf{\emph{fase
di Wahnstimmung}}, in cui si osserva una modificazione della coscienza
dell'Io e del sentimento di giudizio di realtà, che sono alla base della
sicurezza soggettiva ed intersoggettiva dell'Io. Si tratta di una
esperienza indescrivibile ed incomunicabile, dove la perplessità, la
preoccupazione e talora il terrore dominano il soggetto che avverte il
dissolversi dei comuni punti di riferimento che lo legavano al mondo.

Le manifestazioni più frequenti ed importanti nella struttura del
delirio primario sono le seguenti:

\begin{itemize}
\item
  \textbf{Percezione Delirante}, cioè una percezione di per sé corretta
  alla quale viene attribuito un valore semantico abnorme nel senso
  dell'autoriferimento. Le successive elaborazioni comprensibili su base
  psicogenetica, che si sviluppano a partire dalla percezione delirante,
  prendono il nome di \emph{interpretazioni deliranti}.
\item
  \textbf{Intuizione Delirante}, cioè una convinzione che insorge
  improvvisamente e con assoluta certezza soggettiva.
\item
  \textbf{Rappresentazione Delirante}, ovvero una rappresentazione o
  un'immagine mnesica che come la percezione delirante viene investita
  di un significato abnorme.
\end{itemize}

I \textbf{deliri secondari}, o \textbf{deliroidi}, sono derivabili
psicologicamente da un dato psicologico abnorme, e \emph{conseguono ad
un'alterazione dell'affettività}, \emph{sviluppandosi in personalità
psicopatiche come risposta ad eventi traumatici} particolarmente
intensi, situazioni ambientali o fenomeni psicosensoriali abnormi.

In base alle caratteristiche del contenuto deliranti, si possono poi
identificare diversi tipi di delirio:

\begin{itemize}
\item
  \textbf{Delirio di Persecuzione}, pressoché ubiquitario nella
  patologia psichiatrica, potendosi manifestare in tutte le psicosi, e
  all'interno di questo tipo possiamo identificare anche alcune
  sottoforme, come il \emph{delirio di riferimento}, in cui il soggetto
  vive le situazioni e gli avvenimenti come riferiti in modo specifico a
  lui, il \emph{delirio di nocumento}, in cui tutte le persone
  osteggiano il paziente con modalità che nemmeno lui sa precisare bene,
  ed il \emph{delirio di persecuzione propriamente detto}, in cui il
  soggetto crede di essere perseguitato in maniera più o meno manifesta
  da chi gli sta attorno.
\item
  \textbf{Delirio di Trasformazione}, che comprende il \emph{delirio di
  trasformazione cosmica}, quello \emph{palignostico}, quello
  \emph{metempsicotico}, il \emph{delirio zooantropico} e quello
  \emph{ipocondriaco}.
\item
  \textbf{Delirio Mistico}, in cui il paziente sperimenta contatti con
  Dio o si identifica con una divinità.
\item
  \textbf{Delirio Depressivo}, che comprende tematiche connesse
  all'emergenza delle angosce esistenziali primordiali, e al cui interno
  troviamo i \emph{deliri di colpa}, \emph{di rovina} e \emph{di
  ipocondria}, che possono associarsi tra loro dando il cosiddetto
  \emph{delirio nichilistico} o la \emph{sindrome di Cotard}.
\item
  \textbf{Delirio di Grandezza}, che può manifestarsi in forme
  relativamente modeste, con ipervalutazione delle proprie capacità
  fisiche e psichiche, mantenendo un certo rapporto con la realtà ed
  articolandosi nell'ambito del possibile, come nel caso del
  \emph{delirio ambizioso} o del \emph{delirio} \emph{erotomanico},
  oppure si possono avere forme estreme, con un paziente che è
  completamente disancorato dal reale, dominato dalla fantasticheria,
  con un massimo grado di egocentrismo ed autoesaltazione, come nel caso
  del \emph{delirio megalomanico}, del \emph{delirio di potenza}, quello
  \emph{genealogico}, quello \emph{inventario} ed il \emph{delirio di
  enormità}, essendo peraltro comune nella mania delirante, nei disturbi
  deliranti cronici e nelle psico-organiche.
\item
  \textbf{Delirio di Gelosia}, in cui si ha l'assoluta convinzione di
  essere traditi dal proprio partner. Il paziente cerca quindi in tutti
  i modi di trovare le prove dell'inganno sino ad arrivare a cercare di
  estorcere la confessione dell'infedeltà con la violenza. È più
  frequente nel sesso maschile ed è tipico della paranoia alcolica,
  forse legata alla diminuzione della potenza sessuale per l'effetto
  tossico dell'alcol ed il rifiuto da parte del coniuge di avere
  rapporti sessuali.
\end{itemize}

\textbf{Intelligenza}

È la \emph{capacità generica di utilizzare}, in modo adeguato allo
scopo, \emph{tutti gli elementi del pensiero necessari a riconoscere,
impostare e risolvere adeguatamente i nuovi problemi}. La psicopatologia
dell'intelligenza si occupa principalmente di una \textbf{diminuzione
quantitativa dell'intelligenza stessa}, in particolare le \emph{sindromi
da deficit dell'intelligenza} sono condizioni caratterizzate da un
disturbo primario dell'intelligenza dovuto ad un'alterazione organica
cerebrale, sia essa congenita, ereditaria o acquisita. In questi casi il
termine primario nella distinzione tra le sindromi deficitarie
propriamente dette ed i disturbi che accompagnano gli stati confusionali
o determinate condizioni psicopatologiche di tipo funzionale, come gli
stati depressivi.

Le \textbf{\emph{oligofrenie}}, o \textbf{\emph{insufficienze mentali}},
sono sindromi caratterizzate da un \emph{deficit stabile
dell'intelligenza}, dipendente da un \emph{mancato sviluppo delle
capacità intellettive per cause organiche} che abbiano agito in epoca
prenatale, perinatale o postnatale, comunque in un'età della vita in cui
la maturazione cerebrale ed il corrispondente sviluppo psichico sono
ancora in evoluzione.

Le \textbf{\emph{demenze}}, invece, sono sindromi caratterizzate da un
\emph{disturbo, globale e progressivo, dell'intelligenza, che insorgono
quando le capacità intellettive dell'individuo si sono ormai
completamente sviluppate, cioè in età adulta}. In altre parole, la
condizione demenziale esprime un deficit da regressione o decadimento di
funzioni intellettive in precedenza integre, e tale deficit è sempre
riferibile a cause organiche.

\textbf{Affettività}

In base all'intensità, alla durata ed alla modalità di insorgenza i
fenomeni affettivi si dividono in:

\begin{itemize}
\item
  \textbf{\emph{Emozioni}}, cioè stati affettivi quasi sempre reattivi e
  molto intensi, ad insorgenza acuta e rapido esaurimento, oltre ad
  influenzare i processi psichici ed il comportamento, si esprimono sul
  versante neurovegetativo con rossore o pallore del volto, tremore,
  tachicardia ed altri.
\end{itemize}

\begin{itemize}
\item
  \textbf{\emph{Sentimenti}}, che esprimono la particolare risonanza
  affettiva con la quale l'individuo vive la realtà corporea, i suoi
  processi psicologici, la sua socialità ed sono piuttosto persistenti.
\item
  \textbf{\emph{Umore}}, è la tonalità, il ``colorito affettivo'' vitale
  che condiziona permanentemente in un modo o in un altro la nostra
  esistenza. È lo stato basale dell'affettività ed esprime sia il
  temperamento, sia un temporaneo stato affettivo risultante di tutti i
  fenomeni affettivi.
\end{itemize}

Dal punto di vista della psicopatologia dell'umore, questa include
diverse forme:

\begin{itemize}
\item
  \textbf{Umore Depressivo o Depressione}: Consiste in un abbassamento
  del tono dell'umore che si manifesta con sentimenti di tristezza,
  abbattimento, pessimismo e dolore. Nelle forme più gravi tutta la
  corporeità è investita e la vitalità compromessa, con un senso di
  oppressione, malessere, estrema faticabilità, con un vissuto di
  disperazione e di sofferenza morale. L'ideazione è rallentata e povera
  di contenuto, i movimenti volontari sono sempre più lenti, ed è
  ovviamente una forma tipica degli stati depressivi.
\item
  \textbf{Stato Ipertimico}: Si tratta di un innalzamento del tono
  dell'umore che può andare da semplice euforia ad intensa esaltazione.
  Nell'euforia l'umore è gioioso, i sentimenti ed i pensieri hanno una
  tonalità piacevole, il corpo è vissuto con un senso soggettivo di
  benessere. Nella mania c'è accelerazione ideica, logorrea, eccitazione
  psicomotoria, mentre i sentimenti vitali sono esaltati, non esistono
  coercizioni, tutto è possibile. Nelle forme più accentuate si può
  arrivare a gravi accessi d'ira con crisi di furore pantoclastico.
\item
  \textbf{Umore Irritabile}: Condizione di abnorme risonanza affettiva a
  stimoli e situazioni di scarsa entità, con abbassamento della soglia
  emotiva, che si esprime con condizioni di stizza, rabbia o ira,
  associandosi spesso anche a fenomeni neurovegetativi. L'umore
  irritabile si ritrova nelle fasi iniziali della schizofrenia, nella
  mania, negli psicopatici instabili ed impulsivi, nell'oligofrenia e
  nelle psicosi organiche, dove è espressione del deficit intellettivo.
\item
  \textbf{Labilità Affettiva}: Marcata instabilità del tono dell'umore
  che passa in modo improvviso dalla gioia alla tristezza o allo stato
  basale, in risposta a stimoli del tutto inadeguati, anche più volte
  nel corso della stessa giornata, manifestandosi con crisi di riso o
  pianto. È un fenomeno normale in adolescenza, si riscontra nelle
  persone immature ma soprattutto nelle sofferenze organiche (riso e
  pianto spastico).
\item
  \textbf{Paralisi Acuta del Sentimento}: È una condizione conseguente a
  traumi psichici molto intensi, si caratterizza per il vuoto affettivo
  e l'indifferenza momentanea per la situazione a cui il soggetto non sa
  reagire in modo adeguato.
\item
  \textbf{Dissociazione Affettiva}: Indica inadeguatezza o discordanza
  della affettività alla situazione oggettiva. Il grado massimo si
  raggiunge nella schizofrenia, dove una notizia tragica può evocare
  risposte emotive di polarità opposta (\emph{paratimia}).
\item
  \textbf{Apatia}: È una condizione di mancanza di sentimenti che si
  manifesta con distacco e indifferenza affettiva di fronte a qualsiasi
  stimolo. Nell'apatia manca sempre lo stimolo ad agire, e questo
  torpore e congelamento dell'affettività più riscontrarsi in via
  transitoria a seguito di traumi cranici o affezioni cerebrali
  organiche, manifestandosi in modo più marcato e stabile nelle
  schizofrenie e nelle demenze senili.
\item
  \textbf{Ambivalenza Affettiva}: Si tratta di una coesistenza di
  sentimenti opposti nei confronti della stessa persona, cosa o
  situazione. In condizioni normali, quando si provano affetti
  contraddittori verso qualsiasi ``oggetto'', questi influiscono l'uno
  sull'altro portando ad un atteggiamento affettivo complessivo che
  rappresenta la risultante di una serie di valutazioni contrastanti.
  Nella schizofrenia, i sentimenti contrastanti convivono
  simultaneamente, senza che si attenuino o si influenzino a vicenda,
  mentre nelle nevrosi una delle due tendenze viene rimossa, continuando
  però ad influenzare in modo inconscio il paziente.
\item
  \textbf{Anedonia}: Nota anche come ``\emph{sentimento di mancanza di
  sentimento}'' secondo Schneider, è una condizione che si distingue
  dall'apatia per la penosa sensazione di insensibilità che il soggetto
  avverte e di cui si rammarica. Non vengono più provati i comuni moti
  affettivi per i propri familiari, la gioia per la vita o anche il
  dolore, e tutto ciò è motivo di immensa sofferenza emotiva per il
  paziente. si ritrova nella depressione endogena, associato a
  sentimenti di colpa, nelle fasi iniziali di certe forme di
  schizofrenia, e nei disturbi di personalità.
\item
  \textbf{Fobia}: È una condizione di timore o paura immotivata per
  situazioni, oggetti, azioni, associata ad una notevole componente
  ansiosa. Il paziente, pur criticando tali esperienze come abnormi ed
  assurde non riesce a superarle, ed attua una serie di condotte
  tendenti all'evitamento della situazione scatenante. A seconda
  dell'oggetto o situazione che le caratterizza possiamo parlare di
  agorafobia, la claustrofobia e l'ereutofobia (fobia di arrossire).
\item
  \textbf{Ansia}: È uno stato di inquietudine, di pericolo imminente e
  indefinibile. Si associa ad un sentimento di incertezza ed impotenza.
  Mentre la paura è una risposta emozionale ad una minaccia reale,
  l'ansia è priva di un oggetto scatenante o meglio, questo non viene
  chiaramente riconosciuto dall'individuo. All'ansia si associa un
  corteo di sintomi neurovegetativi (tachicardia, ipertensione,
  tachipnea, midriasi, sudorazione, tremori e turbe dell'apparato
  digerente), che sono secondari alle turbe affettive e si ripercuotono
  sul vissuto emozionale, esacerbandolo.
\end{itemize}

\textbf{Istinto}

L'\textbf{\emph{istinto}} viene generalmente definito come un
\emph{bisogno fondamentale che tende al raggiungimento di un fine}, per
cui si può dire che l'istinto tende sempre verso un oggetto per il
raggiungimento del suo scopo. All'istinto è poi correlata una spinta,
detta ``\textbf{\emph{pulsione}}'', che Freud definiva come una forza di
cui si suppone l'esistenza dietro alle tensioni inerenti ai bisogni
dell'organismo. Le influenze intellettive, sociologiche e culturali
agiscono poi plasmando e modificando i comportamenti istintivi, in
quanto tutte le azioni umane dimostrano la compartecipazione maggiore o
minore di attività psichiche più elevate di quelle istintive.

Dal punto di vista della psicopatologia, le turbe dell'istintualità si
manifestano nell'\emph{anoressia nervosa} (in cui c'è un'alterazione
dell'istinto di alimentarsi), nei \emph{disturbi della sessualità} e in
alcune forme di \emph{suicidio} (viene meno l'istinto di sopravvivenza).

\textbf{Volontà (Psicomotilità)}

L'uomo ha la \emph{possibilità di decidere quali tendenze, per lo più
istintuali, ostacolare e quali lasciare che si realizzino, basandosi su
valutazioni di ordine affettivo e cognitivo}, e questo atto mentale
prende il nome di \textbf{\emph{volontà}}. L'atto volontario è sentito a
livello cosciente come scelta consapevole in cui l'Io si realizza in
piena autonomia, e la consapevolezza oggettiva della scelta è ciò che
connota psicologicamente l'attività volontaria.

Anche in questo caso, all'interno della volontà, si possono riconoscere
diverse forme psicopatologiche:

\begin{itemize}
\item
  \textbf{Abulia}, che consiste in un'\emph{inibizione della volontà}, a
  seguito della quale il soggetto è \emph{incapace di prendere una
  decisione}, in merito a circostanze anche banali, oppure di attuare la
  soluzione presa, ed è tipica delle sindromi depressive, nelle
  psiconevrosi ossessivo-compulsive e nelle schizofrenia.
\item
  \textbf{Impulsività}, cioè una spinta irrefrenabile, non mediata dalla
  riflessione, in base alla quale il soggetto compie atti che spesso
  arrecano danno a sé stesso o agli altri, potendosi riscontrare nella
  schizofrenia catatonica e nelle reazioni psicogene, e le cui forme più
  comuni son la piromania, la cleptomania, la dipsomania e le fughe
  impulsive, mentre altre forme sono anche il gioco d'azzardo e la
  tossicomania.
\item
  \textbf{Aumento dell'Attività Motoria}, in cui si passa dai gradi più
  lievi dell'irrequietezza fino ai gradi massimi di iperattività motoria
  con l'eccitamento psicomotorio, che si può manifestare nella bizzarra
  forma di eccitamento maniacale, nella scarica motoria incontrollata
  della schizofrenia catatonica o nelle manifestazioni motorie
  scoordinate ed inconcludenti tipiche delle demenze.
\item
  \textbf{Rallentamento Psicomotorio}, che si manifesta con un
  rallentamento globale dell'attività motoria, come la mimica, la
  gestualità e l'eloquio, che appaiono lente, appesantite e
  difficoltose. Si tratta di una condizione tipica degli stati
  depressivi e di alcuni stati confusionali.
\item
  \textbf{Arresto Psicomotorio}, è il grado estremo di inibizione
  dell'attività motoria, ed è noto anche come stupor. Il soggetto
  affetto non risponde ad alcuno stimolo, è indifferente a quanto accade
  nell'ambiente, giace immobile, conservando piena lucidità di
  coscienza. Si possono distinguere uno stupor melanconico, presente
  nelle forme più gravi di depressione, uno stupor emotivo, che si
  riscontra nell'isteria ed uno stupor catatonico presente nella
  schizofrenia catatonica. Lo stupor catatonico, nello specifico,
  insorge in modo brusco e può durare da pochi minuti ad alcuni mesi ed
  è spesso intervallato da brevi e violente fasi di eccitamento. Il
  soggetto assume atteggiamenti posturali, talora inusuali, che vengono
  mantenuti per tutta la durata dell'episodio, condizione nota coma
  immobilità statuaria. Durante lo stupor catatonico si può verificare
  il fenomeno della catalessia, in cui i malati plasticamente assumono e
  fissano posizioni sia spontanee che passivamente imposte,
  conservandole a lungo anche se innaturali o scomode.
\item
  \textbf{Stereotipie}, sono frammenti di attività motoria e di
  espressione verbale che si ripetono sempre con gli stessi aspetti
  fenomenici per lunghi periodi di tempo. In genere non c'è alcuna
  correlazione con la situazione attuale del soggetto, e tali fenomeni
  possono riguardare atteggiamenti, posizioni, azioni, espressioni
  verbali ed anche la scrittura.
\item
  \textbf{Manierismi}, sono modalità di espressione del linguaggio e dei
  movimenti eccentriche ed artificiose, che rappresentano una caricatura
  di atteggiamenti normali. È una modalità con cui più o meno
  consapevolmente viene perseguito lo scopo di esprimere dei sentimenti
  inesistenti nel soggetto.
\item
  \textbf{Negativismo} e \textbf{Mutacismo}, che può essere attivo o
  passivo; nel negativismo attivo il soggetto compie azioni contrarie a
  quelle richieste, mentre nel negativismo passivo si ha solo una
  resistenza ai comandi o ai tentativi di immobilizzazione. Il
  mutacismo, invece, indica un ostinato rifiuto di rispondere a
  qualsiasi domanda.
\item
  \textbf{Automatismo}, che consiste nell'esecuzione passiva di comandi
  e suggerimenti anche banali ed illogici. Alcune manifestazioni di
  automatismo sono l'\emph{ecoprassia} (imitazione dei gesti),
  l'\emph{ecomimia} (imitazione dell'espressione del volto),
  l'\emph{ecolalia} (imitazione di parole udite) ed \emph{ecografia}
  (ripetizione di scritti).
\end{itemize}

\end{document}
